\chapter{Future work}\label{chapter:future-work}

This master thesis was primarily focused on implementing a prototypical micro-frontend architecture to replace AGnet's internal management software sometime in the future. The process examined whether using GraphQL with Apollo Client can bring performance improvements inside a micro-frontend architecture. The improvements with GraphQL should be implemented with two methods. First, a caching layer should be shared across all micro-frontends. Second, the GraphQL layer should remove fields from queries that are already stored inside the cache. Using a single cache instance for all micro-frontends leads to significant improvements regarding request size and response size. The second method of improving performance by reducing the fields of a query was auspicious, but in the prototype scenario, it did not bring significant improvements. The screen design and structure of the prototype follow the approach of a list view showing a table with some fields of every type. Moreover, a detail for every table row view mainly loads all fields of the type; however, the table view does not prefetch and store enough fields in the cache to make a difference in request and response size. Nevertheless, other applications with a different design might benefit from reducing queries with existing data. The following section explains a theoretical example of an application that might benefit from query reduction.

\subsubsection{Example of an e-commerce platform benefiting from query reduction and a shared cache layer}

This section describes a fictional application that might improve greatly from using query reduction. The described application is a simple e-commerce platform. The e-commerce application has multiple micro-frontends, where each is responsible for a different part of the shopping experience. 

that has a list view of products and a detail view of a product. The list view shows the name, price and a short description of the product. The detail view shows the name, price, description, the category and the manufacturer of the product. The application has a GraphQL layer that fetches the data from a backend. The GraphQL layer is shared across all micro-frontends. The list view and the detail view are two different micro-frontends. The list view fetches the name, price and description of the product. The detail view fetches the name, price, description, category and manufacturer of the pro

\subsubsection{Further prototype development}

The current focus is the further development of micro-frontend architecture. The prototypical implementation, done in this master thesis, has implemented only a small subset of the functional requirements of the overall system. However, the basic setup that all micro-frontends must follow is defined and easily accessible. Creating, integrating, and configuring a new micro-frontend are also well-defined and follow a standardized principle. The process of creating and configuring a new micro-frontend might be automated in the future with the help of Angular's Schematics\footnote{https://angular.io/guide/schematics}.
