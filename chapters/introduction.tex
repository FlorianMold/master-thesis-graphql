\chapter{Introduction}\label{chapter:introduction}

The company AGnet\footnote{\url{https://www.agnet.at/}} has an outdated monolithic system to manage its customers, sales, and so on. The technology stack must be updated, so a migration to newer technology is necessary. The new architecture should consist of micro-frontends and microservices. Microservices are already in development, and a prototypical micro-frontend architecture should be developed in the course of this master thesis. Micro-frontends have different problems than traditional frontend applications, and the developed prototype should tackle the problems that a distributed architecture brings to the table. The prototype should be able to replace the old monolithic application in the future. The prototype should be able to provide the same functionality as the old application.

% motivation weiter ausbauen und schreiben warum queries reduced werden sollen. woher kommt die idee.

\bigskip

\noindent The work is structured in a theoretical part, and a practical part follows it. In the final chapters, the results of the work are presented and discussed. This Chapter describes the motivation for the hypothesis for this thesis. Chapter \ref{chapter:background} describes the principles of software monoliths, \ac{API} abstractions, micro-frontends, and GraphQL. Chapter \ref{chapter:applied-methods} covers the methods applied to the micro-frontend prototype to deal with over-fetching and over-requesting. It describes the most significant problems and hurdles during the development and the solutions. The results can be found in Chapter \ref{chapter:results}, and the discussion revolving around the result is in Chapter \ref{chapter:discussion}. The final Chapter \ref{chapter:conclusion} concludes the thesis. Finally, Chapter \ref{chapter:future-work} gives an outlook for planned and possible extensions in the future.

\section{Motivation}\label{section:introduction:motivation}

The motivation behind this thesis is the creation of a micro-frontend prototype that should replace the old monolithic application within the company. The prototype should be more or less equal to the old application in terms of functionality. The introduction of microservices in the company comes with other problems than a monolithic frontend. For example, all micro-frontends could potentially request the authenticated user. This leads to increased network traffic in comparison to a traditional frontend monolith.

\bigskip

\noindent These problems should be researched and solved by using GraphQL. Many GraphQL clients provide some form of caching. Caching should be used to provide a single caching layer for all micro-frontends to avoid multiple requests to the same resource. GraphQL offers the possibility that the client directly writes the query to the backend. This opens the possibility of removing fields from GraphQL queries that are already in the cache. These approaches to improving performance should be evaluated and compared to the naive approach without any optimizations.

\section{Hypothesis}\label{section:introduction:hypothesis}

The first hypothesis focuses on the performance improvement that GraphQL can bring to micro-frontend architectures.

\paragraph{Hypothesis 1} 
A micro-frontend architecture using a shared GraphQL caching layer with query reduction can solve the problems of over-fetching and over-requesting inside a distributed architecture. The total number of network requests and network traffic can be drastically reduced.

\bigskip

\noindent The second hypothesis suggests that the caching strategy proposed in the thesis is adaptable and flexible enough to be applied successfully in various contexts, regardless of the specific technologies used. The strategy is not limited to a technology stack and can be applied to various systems with different architectures. The thesis focuses on the caching strategy rather than the technologies used, indicating that the proposed approach can be implemented across different technologies and systems.

\paragraph{Hypothesis 2}
The caching strategy proposed in this thesis is versatile enough to be effectively implemented in various contexts, providing sufficient flexibility to be technology agnostic.

\bigskip

\noindent The work results prove that GraphQL can solve the performance problems of a micro-frontend architecture. The proof is provided by designing and implementing a micro-frontend architecture, writing a shared caching layer, and implementing a mechanism to remove fields from a query that are already stored in the cache.
