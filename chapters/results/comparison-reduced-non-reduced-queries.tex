\section{Comparison between queries and reduced queries}

This section takes a closer look at the difference between the request- and response-sizes of the original and reduced queries. The table \ref{table:code:comparison-user-reduction} shows the difference in the size of the user-detail query between the original and reduced queries. The user-detail query and which fields are removed are explained in more detail in section \ref{subsection:background:graphql:example-reduction}.
The size of the GraphQL queries was reduced by half. By removing 8 fields from the original GraphQL query, the size of the network requests to the GraphQL \ac{API} were reduced by about 30\% or by 161 bytes. When retrieving 10 users a total of 1.61 KB can be saved. The response size was reduced by 42\% on average. A total of 2.55 KB in response size is saved if 10 users are fetched from the GraphQL \ac{API}.

\ifshowTables
\begin{table}[H]
  \begin{tabular}{|l|l|l|l|l|}
  \hline
  Query & Request Diff (\%) & Request Diff (B) & Response Diff (\%) & Response Diff (B) \\
  \hline
  User & 30\% & 161 & 42\% & 257 \\
  \hline
  User & 30\% & 161 & 42\% & 257 \\
  \hline
  User & 30\% & 161 & 41\% & 257 \\
  \hline
  User & 30\% & 161 & 42\% & 244 \\
  \hline
  User & 30\% & 161 & 43\% & 251 \\
  \hline
  User & 30\% & 161 & 42\% & 271 \\
  \hline
  User & 30\% & 161 & 42\% & 249 \\
  \hline
  User & 30\% & 161 & 41\% & 263 \\
  \hline
  User & 30\% & 161 & 41\% & 248 \\
  \hline
  User & 30\% & 161 & 41\% & 252 \\
  \hline
  \hline
  \textbf{AVG} & \textbf{30\%} & - & \textbf{42\%} & -  \\
  \hline
  \hline
  \textbf{SUM} & - & \textbf{1610 (1.61 KB)} & - & \textbf{2549 (2.55 KB)} \\
  \hline
  \multicolumn{5}{l}{16 fields, 8 Fields Removed, 8 remaining}
  \end{tabular}
  \caption{Comparison of the user-detail query in request- and response-sizes.}\label{table:code:comparison-user-reduction}
\end{table}
\fi

\noindent The table \ref{table:code:comparison-contract-reduction} shows the difference between original and reduced queries for the contract-detail query. The query fetches the data to display a list of all contracts. By running the query for all contracts 10 fields from the 17 fields are removed from the query. Therefore only 7 of the 17 fields are remaining inside the query, because 58\% were removed. The network requests to the GraphQL \ac{API} can be reduced by an average of 38\% or by 207 bytes. By using the application and querying 10 detail views a contract 2.07 KB of request size can be saved. The response size from the GraphQL \ac{API} can be reduced by about 58\% or 3.96 KB.

\ifshowTables
\begin{table}[H]
  \begin{tabular}{|l|l|l|l|l|}
  \hline
  Query  & Request Diff (\%)  & Request Diff (B) & Response Diff (\%) & Response Diff (B)  \\
  \hline
  Contract & 38\% & 207 & 58\% & 394 \\
  \hline
  Contract & 38\% & 207 & 58\% & 378 \\
  \hline
  Contract & 38\% & 207 & 59\% & 403 \\
  \hline
  Contract & 38\% & 207 & 60\% & 408 \\
  \hline
  Contract & 38\% & 207 & 60\% & 409 \\
  \hline
  Contract & 38\% & 207 & 57\% & 370 \\
  \hline
  Contract & 38\% & 207 & 58\% & 393 \\
  \hline
  Contract & 38\% & 207 & 58\% & 407 \\
  \hline
  Contract & 38\% & 207 & 59\% & 400 \\
  \hline
  Contract & 38\% & 207 & 58\% & 389 \\
  \hline
  \hline
  \textbf{AVG} & \textbf{38\%} & - & \textbf{58\%} & - \\
  \hline
  \hline
  \textbf{SUM} & - & \textbf{2070 (2.07 KB)} & - & \textbf{3951 (3.95 KB)} \\
  \hline
  \multicolumn{5}{l}{17 fields, 10 Fields Removed, 7 remaining}
  \end{tabular}
  \caption{Comparison of the contract-detail query in request- and response-sizes.}\label{table:code:comparison-contract-reduction}
\end{table}
\fi