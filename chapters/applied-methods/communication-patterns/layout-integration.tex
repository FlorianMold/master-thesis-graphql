\subsection{Layout integration}\label{subsection:applied-methods:communication-patterns:layout-integration}

At first, it was troublesome to integrate the remote module into the layout. Therefore the layout service was created with the \texttt{forRoot} and \texttt{forChild} pattern. This pattern typically provides different configuration options for a feature module.

\bigskip

\noindent Here's how it works:

\begin{itemize}
  \item The \texttt{forRoot} function is typically called in the application root module to provide configuration options for a global feature module to the entire application. This method usually returns a module with all providers available globally.
  \item The \texttt{forChild} function is called in a child module that imports the feature module to provide configuration options that are specific to that module. This method usually returns a module with providers only available to components and services within that child module.
\end{itemize}

\noindent The listing \ref{code:applied-methods:communication-patterns:layout-module} shows the implementation of the \texttt{forRoot} and \texttt{forChild} pattern inside the layout-module.

\ifshowListings
  \begin{listing}[H]
  \begin{minted}{typescript}
class LayoutModule {
  static forRoot(): ModuleWithProviders<LayoutModule> {
    return {
      ngModule: LayoutModule,
      providers: [LayoutNavigationService, LayoutTemplateService],
    };
  }

  static forChild(): ModuleWithProviders<LayoutModule> {
    return { ngModule: LayoutModule };
  }
}
  \end{minted}
  \caption{The implementation of forRoot and forChild inside the layout module.}\label{code:applied-methods:communication-patterns:layout-module}
  \end{listing}
\fi

\noindent This approach allows the \texttt{LayoutNavigationService}- and \texttt{LayoutTemplateService}-services available in the root injector. These services can be used to register navigation nodes and templates inside the header. The \texttt{LayoutModule.forChild()} can be used where the components and directives of the layout are needed. Providing the \texttt{LayoutModule} in the root module is shown in the listing \ref{code:applied-methods:communication-patterns:importing-the-root-layout-module}

\ifshowListings
  \begin{listing}[H]
  \begin{minted}{typescript}
@NgModule({
  imports: [ LayoutModule.forRoot() ],
})
class ContactCoreModule {}
  \end{minted}
  \caption{Provide the layout services to the root injector.}\label{code:applied-methods:communication-patterns:importing-the-root-layout-module}
  \end{listing}
\fi
