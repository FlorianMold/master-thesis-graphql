\subsection{Layout integration}\label{subsection:applied-methods:communication-patterns:layout-integration}

This section describes how to integrate a micro-frontend into the application shell layout. To provide micro-frontends with the functionality of integrating their navigation into the layout, the \texttt{LayoutModule} was created using the \texttt{forRoot} and \texttt{forChild} pattern to achieve this goal. Here's how the pattern works:

\begin{itemize}
  \item The \texttt{forRoot} function is typically called in the application root module to provide configuration options for the module to the entire application. This method usually returns a module with all providers available globally.
  \item The \texttt{forChild} function is called in a child module that imports the feature module to provide configuration options that are specific to that module. This method usually returns a module with providers only available to components and services within that child module.
\end{itemize}

\noindent The Listing \ref{code:applied-methods:communication-patterns:layout-module} shows the implementation of the \texttt{forRoot} and \texttt{forChild} pattern inside the layout-module.

\ifshowListings
\begin{listing}[H]
  \begin{minted}{typescript}
@NgModule({...})
class LayoutModule {
  static forRoot(): ModuleWithProviders<LayoutModule> {
    return {
      ngModule: LayoutModule,
      providers: [LayoutNavigationService, LayoutTemplateService]
    };
  }

  static forChild(): ModuleWithProviders<LayoutModule> {
    return { ngModule: LayoutModule };
  }
}
  \end{minted}
  \caption{The implementation of \texttt{forRoot} and \texttt{forChild} inside the layout module.}\label{code:applied-methods:communication-patterns:layout-module}
\end{listing}
\fi

\noindent This approach allows the \texttt{LayoutNavigationService}- and \texttt{LayoutTemplateService}-services to be available as a singleton in the root injector. These services are used to register navigation nodes or templates inside the header. The \texttt{LayoutModule.forChild()} function is used where the components and directives of the \texttt{LayoutModule} are needed. To provide the \texttt{LayoutModule.forRoot()} and its service to the root injector import them in a root module, as shown in Listing \ref{code:applied-methods:communication-patterns:importing-the-root-layout-module}.

\ifshowListings
\begin{listing}[H]
  \begin{minted}{typescript}
@NgModule({
  imports: [ LayoutModule.forRoot() ],
})
class ContactCoreModule {}
  \end{minted}
  \caption{Provide the layout services to the root injector.}\label{code:applied-methods:communication-patterns:importing-the-root-layout-module}
\end{listing}
\fi
