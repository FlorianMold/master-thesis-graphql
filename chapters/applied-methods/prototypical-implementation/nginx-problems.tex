\subsection{Deployment problems}\label{subsection:applied-methods:prototypical-implementation:nginx-problems}

Docker containers are used to serve the micro-frontends. To serve the application Nginx is used. The application is built, and the files are copied into the docker image. The dockerfile is shown in Listing \ref{code:applied-methods:prototype-implementation-dockerfile}.

\ifshowListings
  \begin{listing}[H]
  \begin{minted}{dockerfile}
FROM nginx:alpine
COPY nginx/nginx.conf /etc/nginx/nginx.conf
COPY dist/apps/contact/ /usr/share/nginx/html/
    
CMD [
  "/bin/sh", 
  "-c", 
  "envsubst < /usr/share/nginx/html/assets/settings.template.json >" 
  "/usr/share/nginx/html/assets/settings.json && exec nginx -g daemon off;"
]
  \end{minted}
  \caption{The dockerfile for containerizing a micro-frontend.}\label{code:applied-methods:prototype-implementation-dockerfile}
  \end{listing}
\fi

\noindent Module Federation applications expose a file called \texttt{remoteEntry.mjs}, which is consumed by the application shell. However, Nginx currently cannot interpret the mime-type of \texttt{\*.mjs} which would be \ac{JS}. Therefore, it falls back to the default mime-type \texttt{text/plain}, where \ac{JS} is not executed. As a workaround for this problem, a custom \texttt{nginx.conf} has been created that sets the default mime-type to \texttt{text/javascript}. A part of the configuration is shown in Listing \ref{code:applied-methods:prototype-implementation-custom-nginx-conf}.

\ifshowListings
  \begin{listing}[H]
  \begin{minted}{bash}
default_type text/javascript;
  \end{minted}
  \caption{The custom configuration to set the default mime type for all files.}\label{code:applied-methods:prototype-implementation-custom-nginx-conf}
  \end{listing}
\fi

\noindent The custom configuration allows Nginx to correctly load and interpret the remote module files, which it interprets as \ac{JS} files. Setting another default mime type will not break the web server because an Angular application works just with \ac{JS} files.
