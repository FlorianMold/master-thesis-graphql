\subsection{Backend for frontend architecture}\label{subsection:background:prototypical-implemenation:bff}

The \ac{BFF} pattern was implemented using a GraphQL \ac{API}, and Apollo Server was chosen to implement the \ac{API} for the micro-frontends. All micro-frontends use the same GraphQL endpoint to fetch their data. However, multiple GraphQL \acp{API} could also be used. The GraphQL \ac{API} communicates with the cluster of microservices and aggregates the results to be used for the micro-frontends. When initially the GraphQL \ac{API}, the microservices were still under development, but the data structures for the databases were already finalized. Many mock datasets were generated for the schemas and used for implementing the GraphQL \ac{API}. The queries and mutations were implemented by directly reading and writing the mock data. When the microservices are implemented, the parts of the \ac{API} will be changed to directly communicate with the microservices inside the company's cluster.
