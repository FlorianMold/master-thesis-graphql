\subsection{Backend for frontend}\label{subsection:background:prototypical-implemenation:bff}

The \ac{BFF} pattern was implemented using GraphQL. A single \ac{BFF} service is used for all micro-frontends of the architecture. Apollo Server is used as the GraphQL server and was chosen to implement the \ac{API}. Every micro-frontend uses the same \ac{BFF} service, however, multiple GraphQL \acp{API} could be used. Each micro-frontend is a standalone application, therefore it could communicate with multiple GraphQL \acp{API}.
The GraphQL \ac{API} fetches the data from a cluster of microservices and brings the results into the correct shape for the micro-frontends. The GraphQL \ac{API} was developed, while the microservices were still under development. However, the data model was already finalized. Therefore, the GraphQL \ac{API} was implemented using mock data. The mock data was generated using the database schemas of the microservices. The queries and mutations were implemented by directly reading and writing the mock data. Once the microservices are implemented, the GraphQL \ac{API} will be modified to communicate directly with the microservices within the company.
