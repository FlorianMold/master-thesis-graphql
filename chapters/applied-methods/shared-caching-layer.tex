\section{Shared caching layer between the micro-frontends}\label{section:applied-methods:shared-caching-layer}

With the knowledge from the previous section \ref{section:methods:communication-shell-remote} a shared caching layer was implemented. Apollo GraphQL was chosen as GraphQL client for this prototype. It offers the most support for various frameworks and has a large community. The shell application can provide the instance of the GraphQL cache, which the micro-frontends can inject and use, when setting up their client, as seen in figure \ref{code:methods:graphql-client-cache-provider}. When running the micro-frontend in standalone mode, this provider has to be used inside the native application of the micro-frontend.

For example the contact-application provides this object seen in listing \ref{code:methods:graphql-client-cache-provider} in its providers-array inside the core.module.ts. The host-application has the exactly same configuration inside its core.module.ts.

\ifshowListings
\begin{listing}[H]
\begin{minted}{typescript}
@NgModule({
  providers: [
    {
      provide: UI_GRAPHQL_CLIENT_CACHE,
      useValue: new InMemoryCache(),
    },
  ],
})
export class UiContactCoreModule {}
\end{minted}
\caption{Provide the instance of the cache to dependency injection.}\label{code:methods:graphql-client-cache-provider}
\end{listing}
\fi

UI\_GRAPHQL\_CLIENT\_CACHE is an Angular Injection-token that can be used to provide injectable objects that can be used with dependency injection. (TODO: Injection token link)

These provider must only be set inside the providers of the core.module.ts that the native application provides. Otherwise the remote-module would have their own instances of the GraphQL cache and the shared caching-layer would not work.

Every micro-frontend can have their own instance of a GraphQL client. Only the GraphQL cache is shared between the different applications. Therefore, in theory, it is also possible that every micro-frontend consumes a different GraphQL API.

\ifshowListings
\begin{listing}[H]
\begin{minted}{typescript}
 @NgModule({
   providers: [
     {
       provide: UI_GRAPHQL_CLIENT_OPTIONS_CONFIG,
       useValue: {
         shareCache: true,
         persistCache: false,
         useTypePolicies: true,
         typePolicies: UI_CONTACT_APP_TYPE_POLICIES,
       } as UiGraphQLClientOptionsConfig,
     },
   ],
 })
 export class UiContactRemoteCoreModule {}
\end{minted}
\caption{Extra configuration TODO}\label{code:methods:graphql-client-extra-configuration-options}
\end{listing}
\fi

To make it simple for the micro-frontends to use GraphQL, a function was written that creates the GraphQL client, seen in listing \ref{code:methods:graphql-client-creation}.

\ifshowListings
\begin{listing}[H]
\begin{minted}{typescript}
UiGraphQLClientOptionsModule.withConfig('contact-remote-app', {
  provideGraphQLClientOptions: true,
}),
\end{minted}
\caption{Provide the instance of the cache as injectable.}\label{code:methods:graphql-client-creation}
\end{listing}
\fi

The first parameter of the function is a unique name for the GraphQL client instance. The unique name is mostly used for logging purposes. The second parameter is a configuration object whose options can be configured that no client is created.

For example the GraphQL client could be already provided inside the core-module of the shell-application and the remote-application injects that instance of the GraphQL client. These configuration options were largely added to test the shared caching layer with different options.

The final architecture follows the approach that each micro-frontend has a separate GraphQL client and a shared cache. This allows for the most flexibility as the GraphQL client can be configured individually by each micro frontend.


My prototype creates 12 GraphQL clients, if every remote-application is loaded.

\begin{enumerate}
  \item host-native-app-graphql-client
  \item contact-remote-app-graphql-client
  \item sales-remote-app-graphql-client
  \item user-remote-app-graphql-client
  \item address-remote-widget-graphql-client
  \item contact-list-remote-widget-graphql-client
  \item contact-remote-widget-graphql-client
  \item contract-remote-widget-graphql-client
  \item invoice-remote-widget-graphql-client
  \item person-remote-widget-graphql-client
  \item sales-remote-widget-graphql-client
  \item user-remote-widget-graphql-client
\end{enumerate}

