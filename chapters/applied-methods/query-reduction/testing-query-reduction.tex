\subsection{Query reduction testing layer}\label{subsection:applied-methods:query-reduction:testing-query-reduction}

The abstract class, seen in Listing \ref{code:applied-methods:query-reduction:graphql-client} provides an abstraction for the Apollo Client. The base class was created to make a comparison between Apollo Client's default behavior and the improved behavior of reducing queries. The \texttt{watchQuery} and \texttt{mutate} function have the same \ac{API} as Apollo Client's original \texttt{watchQuery} and \texttt{mutate} functions. Two implementations (\texttt{ReduceGraphQLClientImpl}, \texttt{GraphQLClientImpl}) were created for the base class \texttt{GraphQLClient}. The \texttt{ReduceGraphQLClientImpl} uses query reduction, while \texttt{GraphQLClientImpl} utilizes the original Apollo Client functions. A feature flag is used to determine which implementation is used during runtime.

\ifshowListings
\begin{listing}[H]
\begin{minted}{typescript}
abstract class GraphQLClient {
  abstract watchQuery<TData, TVariables>(
    options: WatchQueryOptions<TData, TVariables>
  ): Observable<QueryResult<TData>>;

  abstract mutate<TData, TVariables>(
    options: MutationOptions<TData, TVariables>
  ): Observable<MutationResult<TData>>;
}
\end{minted}
\caption{The abstract base class for the Apollo Client.}\label{code:applied-methods:query-reduction:graphql-client}
\end{listing}
\fi

\noindent The feature flag is implemented as an injection token, which can be provided in the micro-frontends. The \texttt{REDUCE\_QUERY\_OPTIONS} injection token determines whether query reduction is used. The usage of the injection token is shown in the Listing \ref{code:applied-methods:query-reduction:switch-between-apollo-client-and-query-reduction}. The flag decides whether the \texttt{ReduceGraphQLClientImpl} or \texttt{GraphQLClientImpl} is created. The \texttt{ReduceGraphQLClientImpl} is created when the \texttt{reduceQueries} property is set to \texttt{true}, otherwise the \texttt{GraphQLClientImpl} is created.

\ifshowListings
\begin{listing}[H]
\begin{minted}{typescript}
const REDUCE_QUERY_OPTIONS = 
  new InjectionToken<ReduceQueryOptions>('reduce-query-options');

@NgModule({
  providers: [{
    provide: REDUCE_QUERY_OPTIONS,
    useValue: { reduceQueries: true },
  }]
})
class ContactCoreModule {}
\end{minted}
\caption{Specify whether queries should be reduced with existing fields inside the cache.}\label{code:applied-methods:query-reduction:switch-between-apollo-client-and-query-reduction}
\end{listing}
\fi

