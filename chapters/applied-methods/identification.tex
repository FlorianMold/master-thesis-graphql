\section{Identification of micro-frontends}\label{section:applied-methods:identification-micro-frontends}

The first step before starting to implement the micro-frontend architecture was to identify the parts of the legacy application that should be extracted into microservices and micro-frontends. The application is a large software monolith with a single database. In collaboration with the product owner, parts of the legacy application were identified as candidates for micro-frontends. 

The user interface of the old implementation is also one large application.

\bigskip

\noindent The complete legacy application can't be prototyped in the course of this master-thesis, therefore the three most important bounded contexts were identified. The bounded contexts are the following:

\begin{itemize}
  \item User
  \item Sales
  \item Contact
  \item Dashboard
\end{itemize}