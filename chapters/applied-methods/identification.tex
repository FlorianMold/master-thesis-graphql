\section{Identifying of micro-frontends}\label{section:applied-methods:identification-micro-frontends}

The first step before starting to implement the micro-frontend architecture was to identify the parts of the legacy application that should be extracted into microservices and micro-frontends. The legacy application is a large software monolith with a single database. In collaboration with the product owner, the parts of the legacy application were identified as candidates for micro-frontends. The user interface of the old implementation is also one large application.

\bigskip

\noindent The complete application can't be prototyped at once, so the first three bounded contexts that were identified are:

\begin{itemize}
  \item User
  \item Sales
  \item Contact
\end{itemize}