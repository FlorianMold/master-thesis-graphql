\section{GraphQL}

GraphQL was developed by Facebook and refers to itself as the query language for an API. It provides the client with the opportunity to ask exactly for the data is needed. GraphQL offers the advantage that all data must be loaded from only one URL. With the help of the types within the GraphQL schema, the technology offers an understandable description of the
API for clients. \cite{misc:-:graphql-org} The functionality of GraphQL on the frontend can be compared to SQL on the database level. The client writes its queries with the desired fields from a dataset.

The official documentation states that GraphQL is a query language for APIs and a runtime for fulfilling those queries with your existing data. GraphQL provides a complete and understandable description of the data in your API, gives clients the power to ask for exactly what they need and nothing more, makes it easier to evolve APIs over time, and enables powerful developer tools. GraphQL is implemented in many languages and frameworks. It has a large ecosystem of libraries and tools.

\subsection{Origins and history}

GraphQL was initially developed at Facebook in 2012. In 2015 the project was made open-source and available to the public. The motivation behind the initial development of GraphQL was the limited flexibility of available API technologies like REST.

\subsection{Apollo Server and Client}

GraphQL is a query language and follows specific rules. The development of a GraphQL server and client is up to the application developer. Facebook has developed their own implementation of a GraphQL server and client. The server is called GraphQL.js and the client is called Relay.

Apollo 

Apollo Client is m

\subsubsection{How does the in-memory cache work?}

This section describes how the cache in apollo-client works. By default all GraphQL requests made with \textbf{ApolloClient} are cached inside the browsers memory. This enables Apollo Client to respond almost immediately to queries for already-cached data, without even sending a network request. This is needed, to ensure to reduce round-trips to the server in subsequent requests of the same query, because the requested data can be served from the cache. The caching mechanism reduces the load of the server, but introduces issues with cache management. \textbf{ApolloClient} includes a caching mechanism called ApolloCache and the propretary implementation called \textbf{InMemoryCache}. There are several Open-Source alternatives, that implement the \textbf{ApolloCache} interface.

\ifshowImages
\begin{figure}[!htbp]
\centering
\includegraphics[width=0.6\linewidth]{images/background/apollo/apollo-client-basic-cache.jpeg}
\caption{All requests made during the measurement of the first approach}\label{figure:background:user-query-first-time}
\end{figure}
\fi

The flow of the cache, when the query \textbf{user} is executed the first time is shown in figure \ref{figure:background:user-query-first-time}.


\ifshowImages
\begin{figure}[!htbp]
\centering
\includegraphics[width=0.6\linewidth]{images/background/apollo/apollo-client-basic-cache-warm.jpeg}
\caption{All requests made during the measurement of the first approach}\label{figure:background:user-query-second-time}
\end{figure}
\fi

When the query is executed later with the same parameters, the flow looks like in figure \ref{figure:background:user-query-second-time}.

In order to correctly understand cache updates, it is important to understand the structure of the cache. The structure of the \textbf{InMemoryCache} is a simple normalized object. When the cache is empty, it is just an empty object. When the following query is requested from the server and the response is stored in the cache, the cache will look like this:

\ifshowListings
\begin{listing}[H]
\begin{minted}{typescript}
query {
  users {
    id
    username
    email
  }
}
\end{minted}
\caption{An example of a query}\label{code:background:query-user-cache}
\end{listing}
\fi

And the server responds with the following result. The \textbf{\_\_typename} property is automatically appended to the query by the \textbf{ApolloClient}.

\ifshowListings
\begin{listing}[H]
\begin{minted}{typescript}
{
  users: [
    {
      __typename: 'User',
      id: '36bad921-8fcf-4f33-9f29-0d3cd70205c8',
      username: 'Florian',
      email: 'florian@test.io'
    }, 
    {
      __typename: 'User',
      id: 'a2096556-9a4e-4994-9de8-86c9e85ed6a1',
      username: 'Daniel',
      email: 'daniel@test.io'
    }
  ]
}
\end{minted}
\caption{The result of the GraphQL query from listing \ref{code:background:query-user-cache}}\label{code:background:query-user-response-result}
\end{listing}
\fi

The \textbf{ApolloClient} will update the cache so that it looks like this.

\ifshowListings
\begin{listing}[H]
\begin{minted}{typescript}
{
  ROOT_QUERY: {
    __typename: 'Query',
    users: [
      { __ref: 'User:36bad921-8fcf-4f33-9f29-0d3cd70205c8', },
      { __ref: 'User:a2096556-9a4e-4994-9de8-86c9e85ed6a1', },
    ],
  },
  'User:36bad921-8fcf-4f33-9f29-0d3cd70205c8': {
    __typeName: 'User',
    id: '36bad921-8fcf-4f33-9f29-0d3cd70205c8',
    username: 'Florian',
    email: 'florian@test.io',
  },
  'User:a2096556-9a4e-4994-9de8-86c9e85ed6a1': {
    __typeName: 'User',
    id: 'a2096556-9a4e-4994-9de8-86c9e85ed6a1',
    username: 'Daniel',
    email: 'daniel@test.io',
  }
}
\end{minted}
\caption{The data inside the cache with the response from listing \ref{code:background:query-user-response-result}}\label{code:background:query-user-cache-representation}
\end{listing}
\fi

Let's describe how the response from the server is transformed into the representation from the cache seen in listing \ref{code:background:query-user-cache-representation}. The cache object contains a key \textbf{ROOT\_QUERY}. This element contains the name of the queries that were executed and the results from all queries. The query \textbf{allUsers} was fetched, therefore the \textbf{ROOT\_Query} contains a field with the name \textbf{users}. The listing \ref{code:background:query-user-cache-representation} shows that the content of the \textbf{ApolloClient} is clearly different from the servers response. Instead of the user-information every array item consists of an object with a \textbf{\_\_ref} key. The value of the key is simply the \textbf{\_\_typename} and \textbf{id} of the user concatenated. The data from the response has been normalised and added to the cache object. Next to the \textbf{ROOT\_QUERY} element, the actual user-information is stored. Each has the same key as the \textbf{\_\_ref} from the \textbf{ROOT\_QUERY}. 

The same principle applies to arbitrary deep queries. The following query produces:

\ifshowListings
\begin{listing}[H]
\begin{minted}{typescript}
query {
  allUsers {
    id
    username
    Title {
      id
      name
    }
  }
}
\end{minted}
\caption{An example of a query}\label{code:background:nested-query-user-cache}
\end{listing}
\fi

And the server responds with the following result. The \textbf{\_\_typename} property is automatically appended to the query by the \textbf{ApolloClient}.

\ifshowListings
\begin{listing}[H]
\begin{minted}{typescript}
{
  users: [
    {
      __typename: 'User',
      id: '36bad921-8fcf-4f33-9f29-0d3cd70205c8',
      username: 'Florian',
      title: {
        __typename: 'Title',
        id: '2adb1120-d911-4196-ab1b-d5043cc7a00a',
        name: 'BSc.'
      }
    }, 
    {
      __typename: 'User',
      id: 'a2096556-9a4e-4994-9de8-86c9e85ed6a1',
      username: 'Daniel',
      title: {
        __typename: 'Title',
        id: '2adb1120-d911-4196-ab1b-d5043cc7a00a',
        name: 'BSc.'
      }
    }
  ]
}
\end{minted}
\caption{The result of the GraphQL query from listing \ref{code:background:nested-query-user-cache}}\label{code:background:nested-query-user-response-result}
\end{listing}
\fi

\ifshowListings
\begin{listing}[H]
\begin{minted}{typescript}
{
  ROOT_QUERY: {
    __typename: 'Query',
    users: [
      { __ref: 'User:36bad921-8fcf-4f33-9f29-0d3cd70205c8', },
      { __ref: 'User:a2096556-9a4e-4994-9de8-86c9e85ed6a1', },
    ],
  },
  'User:36bad921-8fcf-4f33-9f29-0d3cd70205c8': {
    __typeName: 'User',
    id: '36bad921-8fcf-4f33-9f29-0d3cd70205c8',
    username: 'Florian',
    Title: {
      __ref: 'Title:2adb1120-d911-4196-ab1b-d5043cc7a00a',
    },
  },
  'User:a2096556-9a4e-4994-9de8-86c9e85ed6a1': {
    __typeName: 'User',
    id: 'a2096556-9a4e-4994-9de8-86c9e85ed6a1',
    username: 'Daniel',
    Title: {
      __ref: 'Title:2adb1120-d911-4196-ab1b-d5043cc7a00a',
    },
  }
  'Title:2adb1120-d911-4196-ab1b-d5043cc7a00a': {
    __typeName: 'Address',
    id: '2adb1120-d911-4196-ab1b-d5043cc7a00a',
    name: 'BSc.',
  },
}
\end{minted}
\caption{The data inside the cache with the response from listing \ref{code:background:nested-query-user-response-result}}\label{code:background:nested-query-user-cache-representation}
\end{listing}
\fi

The \textbf{ROOT\_QUERY} is exactly the same as with the query before. Each user contains a reference to a title. Both users have the same title, therefore the server returned duplicate data. But the cache normalisation causes the title to be only present once in the cache.

This behaviour is very helpful, because when a cache item is updated, the entire cache object doesn't have to be traversed in search for the instance that has been changed. Only a single item has to updated.

The cache normalisation works when the query contains either a \textbf{\_id} or \textbf{id} field. Without an id the object can't be normalized. Here is an example:

\ifshowListings
\begin{listing}[H]
\begin{minted}{typescript}
query {
  allUsers {
    id
    username
    Title {
      name
    }
  }
}
\end{minted}
\caption{An example of a query}\label{code:background:no-id-query-user-cache}
\end{listing}
\fi

The response: 

\ifshowListings
\begin{listing}[H]
\begin{minted}{typescript}
{
  users: [
    {
      __typename: 'User',
      id: '36bad921-8fcf-4f33-9f29-0d3cd70205c8',
      username: 'Florian',
      title: {
        __typename: 'Title',
        name: 'BSc.'
      }
    }, 
    {
      __typename: 'User',
      id: 'a2096556-9a4e-4994-9de8-86c9e85ed6a1',
      username: 'Daniel',
      title: {
        __typename: 'Title',
        name: 'BSc.'
      }
    }
  ]
}
\end{minted}
\caption{The result of the GraphQL query from listing \ref{code:background:no-id-query-user-cache}}\label{code:background:no-id-query-user-response-result}
\end{listing}
\fi

And the cache looks like the following:

\ifshowListings
\begin{listing}[H]
\begin{minted}{typescript}
{
  ROOT_QUERY: {
    __typename: 'Query',
    users: [
      { __ref: 'User:36bad921-8fcf-4f33-9f29-0d3cd70205c8', },
      { __ref: 'User:a2096556-9a4e-4994-9de8-86c9e85ed6a1', },
    ],
  },
  'User:36bad921-8fcf-4f33-9f29-0d3cd70205c8': {
    __typeName: 'User',
    id: '36bad921-8fcf-4f33-9f29-0d3cd70205c8',
    username: 'Florian',
    Title: {
      name: 'BSc.',
    },
  },
  'User:a2096556-9a4e-4994-9de8-86c9e85ed6a1': {
    __typeName: 'User',
    id: 'a2096556-9a4e-4994-9de8-86c9e85ed6a1',
    username: 'Daniel',
    Title: {
      name: 'BSc.',
    },
  }
}
\end{minted}
\caption{The data inside the cache with the response from listing \ref{code:background:no-id-query-user-response-result}}\label{code:background:no-id-query-user-cache-representation}
\end{listing}
\fi

This should be avoided. If data of an unnormalised object has to be updated every occurence of the item in the cache has to be updated manually. You must also never request the same thing sometimes with an id and sometimes without, because \textbf{ApolloClient} will throw an error when trying to update the cache after such a query.


The Apollo Client's cache stores the data a flat lookup table of objects that reference eachother. \cite{misc:-:apollo-cache-overview}

Whenever the Apollo Client receives the response data of a query it does the following. a

\begin{enumerate}
  \item \textbf{Identify objects}:
  \item \textbf{Generate cache IDs}:
  \item \textbf{Replace object fields with references}:
  \item \textbf{Store normalized objects}:
\end{enumerate}
