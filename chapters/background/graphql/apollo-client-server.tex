\subsection{Apollo Server and Apollo Client}\label{subsection:background:graphql:apollo-server-client}

GraphQL is a query language and follows the rules of the specification. The application developers are responsible for implementing the server and client architecture, in addition to adhering to the GraphQL specification. Facebook has created its own GraphQL implementation of the specification in \ac{JS} with GraphQL.js\footnote{\url{https://github.com/graphql/graphql-js}} for the backend and Relay\footnote{\url{https://github.com/facebook/relay}} for the frontend. Relay is a prominent example of a GraphQL client, however it is only available for Facebook's React framework. After evaluating multiple GraphQL clients, Apollo Client was chosen because it supports almost every possible programming language and framework and is the most popular client.

\bigskip

\noindent Apollo Server\footnote{\url{https://github.com/apollographql/apollo-server}} is an open-source implementation of the GraphQL specification for the backend, and it offers all features that the GraphQL specification specifies. Moreover, it supports Apollo Sandbox\footnote{\url{https://www.apollographql.com/docs/graphos/explorer/sandbox/}} out of the box. \cite{misc:-:background:graphql:apollo-server-introduction} Apollo Sandbox supports local development. Apollo Sandbox can load the GraphQL Schema from the server with GraphQL's introspection features. \cite{misc:-:background:graphql:apollo-sandbox} The development environment of Apollo Sandbox allows the client to execute GraphQL queries and mutations directly from the browser.

\bigskip

\noindent Apollo Client\footnote{\url{https://github.com/apollographql/apollo-client}} is a community-driven project with npm packages for almost all frontend development environments like Angular, React, and Vue.js. The library fetches, caches, and manages the data from a GraphQL \ac{API} for the application. The dependency for Angular is designed with Angular patterns in mind. It integrates perfectly with the framework. Apollo Client offers the possibility to cache already made requests. \cite{misc:-:background:graphql:apollo-angular-client-overview, misc:-:background:graphql:apollo-client-overview} Caching is especially important as it can reduce the number of roundtrips to the server because duplicate does have to be loaded. The inner workings of the cache to prove or disprove the first hypothesis are explained in the following sections.

\subsubsection{How the in-memory cache works}\label{subsubsection:background:graphql:apollo-server-client:in-memory-cache-working}

This section describes how the caching mechanism of the Apollo Client works. The structure of the cache is a local, normalized, in-memory cache. All GraphQL requests made with Apollo Client are cached inside the browser's memory by default. By utilizing caching, Apollo Client can quickly respond to queries for data that has already been cached, without the need to send a network request. This mechanism is needed to reduce round-trips to the server in subsequent requests of the same query because the requested data can be served from the cache. \cite{misc:-:background:graphql:apollo-client-cache-overview} Caching mechanisms generally reduce the server's load but introduce issues with cache management.

\bigskip

\noindent \textbf{Apollo Client} includes a caching interface called \texttt{ApolloCache} and a proprietary implementation named \texttt{InMemoryCache}. Several Open-Source alternatives implement the \texttt{ApolloCache} interface, but Apollo Client's implementation is well-supported and receives regular updates. Therefore it was chosen for the micro-frontend prototype.

\bigskip

\noindent Figure~\ref{fig:background:graphql:user-query-first-time} shows an exemplary query to fetch a user with its unique id. First, the Apollo Client checks whether the user with the given id has already been stored inside the cache. If the query with that id has yet to be executed, a network query to fetch the user must be made. The result from the GraphQL Server is stored inside the cache and the user is returned to the application. Four steps must be made to render the user data on the screen. \cite{misc:-:background:graphql:apollo-client-cache-overview}

\ifshowImages
\begin{figure}[H]
  \centering
  \includegraphics[width=0.6\linewidth]{images/background/graphql/apollo/apollo-client-basic-cache.png}
  \caption{The execution of a GraphQL query that is not stored in the cache. (Adapted from \cite{misc:-:background:graphql:apollo-client-cache-overview})}\label{fig:background:graphql:user-query-first-time}
\end{figure}
\fi

\noindent If a request with the same user id is made again, the flow of execution looks like in Figure \ref{fig:background:graphql:user-query-second-time}. As seen in the figure, no network request has to be made to the GraphQL \ac{API}, because everything is served from the Apollo Cache alone. Compared to the four steps in Figure \ref{fig:background:graphql:user-query-first-time}, only two steps need to be done here. \cite{misc:-:background:graphql:apollo-client-cache-overview}

\ifshowImages
\begin{figure}[H]
  \centering
  \includegraphics[width=0.7\linewidth]{images/background/graphql/apollo/apollo-client-basic-cache-warm.png}
  \caption{Execute the same query, which reduces the necessary steps. (Adapted from \cite{misc:-:background:graphql:apollo-client-cache-overview})}\label{fig:background:graphql:user-query-second-time}
\end{figure}
\fi

\subsubsection{Data normalization}\label{subsubsection:background:graphql:apollo-server-client:data-normalization}

This section describes what happens when the Apollo Client receives the response from a query. The Apollo Client normalizes the data and stores it inside the cache. The data is stored in a normalized way to avoid duplication. The following paragraphs describe the steps from a GraphQL query to the objects inside the cache.

\bigskip

\noindent To correctly understand cache updates, it is essential to understand the structure of the cache before. The \texttt{InMemoryCache} is a simple normalized \ac{JS} object. The Apollo Client stores the cache data as a flat lookup table of objects referencing each other. An empty cache is just an empty \ac{JS} object. These objects correspond to the objects that are returned by GraphQL queries. A single cache object might include fields fetched by multiple queries, allowing multiple queries to fetch different fields for the same object. \cite{misc:-:background:graphql:apollo-client-cache-overview} The following paragraphs describe the steps from a GraphQL query to the objects inside the cache.

\paragraph{1. Identify objects}\label{paragraph:background:graphql:apollo-server-client:data-normalization:identify-objects}

The cache identifies all distinct objects included in the query response. For example, take the query from Listing \ref{code:background:graphql:query-user-cache}, which returns the \texttt{id}, \texttt{username}, and \texttt{email} of every user.

\ifshowListings
\begin{listing}[H]
  \begin{minted}{typescript}
query {
  users {
    id
    username
    email
  }
}
  \end{minted}
  \caption{GraphQL query that fetches the \texttt{id}, \texttt{username}, and \texttt{email} of every user.}\label{code:background:graphql:query-user-cache}
\end{listing}
\fi

\noindent The server responds to the query with the following object seen in Listing \ref{code:background:graphql:query-user-response-result}. The \texttt{\_\_typename} property is automatically appended to the query by the Apollo Client to identify which type the object has.

\ifshowListings
\begin{listing}[H]
  \begin{minted}{typescript}
{
  users: [
    {
      __typename: 'User',
      id: '36bad921-8fcf-4f33-9f29-0d3cd70205c8',
      username: 'Florian',
      email: 'florian@test.io'
    },
    {
      __typename: 'User',
      id: 'a2096556-9a4e-4994-9de8-86c9e85ed6a1',
      username: 'Lisa',
      email: 'lisa@test.io'
    }
  ]
}
  \end{minted}
  \caption{The result of the GraphQL query from Listing \ref{code:background:graphql:query-user-cache}.}\label{code:background:graphql:query-user-response-result}
\end{listing}
\fi

The caching mechanism identifies the following objects to be cached.

\begin{itemize}
  \item A \texttt{User} with id \texttt{36bad921-8fcf-4f33-9f29-0d3cd70205c8}
  \item A \texttt{User} with id \texttt{a2096556-9a4e-4994-9de8-86c9e85ed6a1}
\end{itemize}

\paragraph{2. Generate cache IDs}\label{paragraph:background:graphql:apollo-server-client:data-normalization:generate-cache-ids}

Once all the objects that should be cached are identified, the caching system generates a unique cache \ac{ID} for each object. The cache \ac{ID} is a unique identifier that is used to identify a specific object that is stored in the \texttt{InMemoryCache}. The cache \acp{ID} for the objects from the previous section are:

\begin{itemize}
  \item \texttt{User:36bad921-8fcf-4f33-9f29-0d3cd70205c8}
  \item \texttt{User:a2096556-9a4e-4994-9de8-86c9e85ed6a1}
\end{itemize}

\noindent The default method for generating a cache \texttt{ID} for an object involves concatenating the object's \texttt{\_\_typename} and \texttt{id} (or \texttt{\_id}) fields. When the cache system is unable to generate a cache ID for an object, that object is stored directly within its parent object. As a result, the object must be accessed and referenced through its parent. As a consequence of storing certain objects within their parent object, the cache may not always be completely flat in its structure. This behavior is shown with the query in Listing \ref{code:background:graphql:no-id-query-user-cache}. 

\ifshowListings
\begin{listing}[H]
  \begin{minted}{typescript}
query {
  allUsers {
    id
    username
    Title {
      name
    }
  }
}
  \end{minted}
  \caption{Fetch the \texttt{id}, \texttt{username} and the \texttt{name} of the title of the user, without the \texttt{id} of the title.}\label{code:background:graphql:no-id-query-user-cache}
\end{listing}
\fi

\noindent The query from the listing \ref{code:background:graphql:no-id-query-user-cache} does not query the \texttt{id} field of the title. Therefore the cache stores the title object directly inside the user object. The result inside the cache is shown in the Listing
\ref{code:background:graphql:no-id-query-user-cache-representation}. 

\ifshowListings
\begin{listing}[H]
  \begin{minted}{typescript}
{
  'User:36bad921-8fcf-4f33-9f29-0d3cd70205c8': {
    __typeName: 'User',
    id: '36bad921-8fcf-4f33-9f29-0d3cd70205c8',
    username: 'Florian',
    Title: { name: 'BSc.' }
  },
  'User:a2096556-9a4e-4994-9de8-86c9e85ed6a1': {
    __typeName: 'User',
    id: 'a2096556-9a4e-4994-9de8-86c9e85ed6a1',
    username: 'Lisa',
    Title: { name: 'BSc.' }
  }
}
  \end{minted}
  \caption{The content of the cache after fetching the query from Listing \ref{code:background:graphql:no-id-query-user-cache}.}\label{code:background:graphql:no-id-query-user-cache-representation}
\end{listing}
\fi

\noindent Omitting object \acp{ID} from a query should be avoided. If data of an un-normalized object should be updated, every occurrence of the item in the cache has to be updated manually. It should be avoided that the same object is queried, sometimes with the id and sometimes without the id because Apollo Client will throw an error when trying to update the cache after such a query.

\paragraph{3. Replace object fields with references}\label{paragraph:background:graphql:apollo-server-client:data-normalization:replace-object-fields-with-references}

After identifying all objects that should be cached, the caching system replaces the value of each field that contains an object with a reference to the corresponding object. Listing \ref{code:background:graphql:nested-query-user-cache} shows a query that demonstrates how the objects from a query are transformed in the \texttt{InMemoryCache} representation.

\ifshowListings
\begin{listing}[H]
  \begin{minted}{typescript}
query {
  allUsers {
    id
    username
    Title {
      id
      name
    }
  }
}
  \end{minted}
  \caption{A GraphQL query to fetch all users.}\label{code:background:graphql:nested-query-user-cache}
\end{listing}
\fi

Listing \ref{code:background:graphql:nested-query-response-user-cache} shows the result from the GraphQL server for the query from Listing \ref{code:background:graphql:nested-query-user-cache}.

\ifshowListings
\begin{listing}[H]
    \begin{minted}{typescript}
{
  __typename: 'User',
  id: '36bad921-8fcf-4f33-9f29-0d3cd70205c8',
  username: 'Florian',
  title: {
    __typename: 'Title',
    id: '2adb1120-d911-4196-ab1b-d5043cc7a00a',
    name: 'BSc.'
  }
}
    \end{minted}
    \caption{The result of the GraphQL query from Listing \ref{code:background:graphql:nested-query-user-cache}.} \label{code:background:graphql:nested-query-response-user-cache}
\end{listing}
\fi

\noindent Listing \ref{code:background:graphql:nested-query-response-after-replacement} shows how the object is stored inside the \texttt{InMemoryCache}. In comparison to the structure of the cache from Listing \ref{code:background:graphql:no-id-query-user-cache-representation}, the \texttt{title} field now references the appropriate normalized \texttt{Title} object. If another \texttt{User} with the same \texttt{title} is stored inside the in-memory cache, that normalized \texttt{Title} object is reused. By using normalization, the amount of duplicated data stored in the cache can be significantly reduced. Additionally, normalization helps ensure that the data in the cache remains in sync with the data stored on the server.

\ifshowListings
\begin{listing}[H]
  \begin{minted}{typescript}
{
  __typename: 'User',
  id: '36bad921-8fcf-4f33-9f29-0d3cd70205c8',
  username: 'Florian',
  title: { __ref: 'Title:36bad921-8fcf-4f33-9f29-0d3cd70205c8' }
}
  \end{minted}
  \caption{The structure of the cache after the user object is stored.}\label{code:background:graphql:nested-query-response-after-replacement}
\end{listing}
\fi

\paragraph{4. Store normalized objects}\label{paragraph:background:graphql:apollo-server-client:data-normalization:store-normalized-objects}

When an incoming object shares the same cache \ac{ID} as an object already stored in the cache, the two objects' fields are merged. If there are any overlapping fields between the two objects, the incoming object's values will overwrite the existing cached values. Any fields that exist in only one of the objects will be preserved. This process of normalization creates a partial copy of the data graph on the client. \cite{misc:-:background:graphql:apollo-client-cache-overview}

\bigskip

\noindent Listing \ref{code:background:graphql:nested-query-user-cache-representation} shows a completely normalized cache. Each user contains a reference to the same title. Therefore the GraphQL \ac{API} returned duplicate data. However, the cache normalization causes the title to be only present once inside the cache. This behavior is helpful because when a cache item is updated, the entire cache does not have to be traversed in search of the instance that has been changed. Only a single item has to be updated. However, the cache normalization only works when the query contains either a \texttt{\_id} or \texttt{id} field.

\ifshowListings
\begin{listing}[H]
  \begin{minted}{typescript}
{
  'User:36bad921-8fcf-4f33-9f29-0d3cd70205c8': {
    __typeName: 'User',
    id: '36bad921-8fcf-4f33-9f29-0d3cd70205c8',
    username: 'Florian',
    Title: { __ref: 'Title:2adb1120-d911-4196-ab1b-d5043cc7a00a' },
  },
  'User:a2096556-9a4e-4994-9de8-86c9e85ed6a1': {
    __typeName: 'User',
    id: 'a2096556-9a4e-4994-9de8-86c9e85ed6a1',
    username: 'Lisa',
    Title: { __ref: 'Title:2adb1120-d911-4196-ab1b-d5043cc7a00a' },
  }
  'Title:2adb1120-d911-4196-ab1b-d5043cc7a00a': {
    __typeName: 'Title',
    id: '2adb1120-d911-4196-ab1b-d5043cc7a00a',
    name: 'BSc.',
  },
}
  \end{minted}
  \caption{The structure of the cache with the result from the query from Listing \ref{code:background:graphql:nested-query-user-cache}.}\label{code:background:graphql:nested-query-user-cache-representation}
\end{listing}
\fi

\subsubsection{Display the structure of the cache}\label{subsubsection:background:graphql:apollo-server-client:understanding-cache-structure}

Apollo Client offers local development tools for the browser in the form of browser extensions. The Apollo Client Developer Tools\footnote{\url{https://chrome.google.com/webstore/detail/apollo-client-devtools/jdkknkkbebbapilgoeccciglkfbmbnfm}} can be installed for Chrome and Firefox. The extension can be found in the Chrome Web Store and the Firefox Add-ons Store. The browser extension adds a section to the browser's developers' tools. \cite{misc:-:background:graphql:apollo-developer-tools} The development tools can be seen in Figure \ref{fig:background:graphql:apollo:apollo-dev-tools}. The figure shows the normalized objects inside the cache.

\ifshowImages
\begin{figure}[H]
  \centering
  \includegraphics[width=1\linewidth]{images/background/graphql/apollo/apollo-dev-tools.jpg}
  \caption{The content of the cache inside the Apollo Client Developer Tools.}\label{fig:background:graphql:apollo:apollo-dev-tools}
\end{figure}
\fi

\noindent The development tools offer the following four main features: \cite{misc:-:background:graphql:apollo-developer-tools}

\begin{itemize}
  \item \textbf{GraphiQL}: To retrieve data, either send a query to the server using the configured Apollo Client instance within the web application or query the Apollo Client cache to check which data is currently loaded.
  \item \textbf{Watched query inspector}: View active queries, variables, cached results, and re-run individual queries.
  \item \textbf{Mutation inspector}: View the active mutations and their variables, and re-run individual mutations.
  \item \textbf{Cache inspector}: You can visualize the Apollo Client and search it by either field name or value, or both.
\end{itemize}

\noindent Another method to access the content of the cache is through the \texttt{window} object in \ac{JS}. The object can be accessed by executing \texttt{window.\_\_APOLLO\_CLIENT\_\_.cache.extract()} inside the browser's development console. The result of this method invocation can be seen in Figure \ref{fig:background:graphql:apollo:apollo-cache-browser-window}.

\ifshowImages
  \begin{figure}[H]
    \centering
    \includegraphics[width=1\linewidth]{images/background/graphql/apollo/apollo-cache-browser-window.jpg}
    \caption{View the content of the cache inside the development tools.}\label{fig:background:graphql:apollo:apollo-cache-browser-window}
  \end{figure}
\fi

\noindent This approach has the benefit, that the content of the cache can be explored with \ac{JS} functions inside the browser development console.

\subsubsection{Type-Policies}\label{subsubsection:background:graphql:apollo-server-client:type-policies}

Type policies are a way to define how the client stores and manages data in the cache. The constructor of the \texttt{InMemoryCache} accepts the configuration for type policies as an object. The \texttt{typePolicies} object the behavior of the cache can be customized on a type-by-type basis. Type policies are fine-grained; they allow to customize how a specific field inside the cache is read and written. A type policy contains multiple field policies that customize the behavior of each field for the type. \cite{misc:-:background:graphql:apollo-client-cache-reading-writing}

\bigskip

\noindent A field policy for a field consists of: \cite{misc:-:background:graphql:apollo-client-cache-reading-writing}

\begin{itemize}
  \item A \texttt{read} function which is called when the field is read from the cache.
  \item A \texttt{merge} function which is called when the field is written to the cache.
  \item The key arguments of a field. These arguments are an array of fields, which identify a field. This so-called \texttt{keyArgs} avoids storing duplicate data in the cache.
\end{itemize}

\paragraph{Read function}

Listing \ref{code:background:graphql:read-type-policy} shows the usage of a \texttt{read} function. The cache executes that function when the client queries for a user's first name. The \texttt{InMemoryCache} then returns the function's response instead of the cached value. The first parameter of the function is the current value of the field. The second parameter is an object that contains the arguments of the field and several helper functions and properties. \cite{misc:-:background:graphql:apollo-client-cache-reading-writing} The \texttt{read} function of Listing \ref{code:background:graphql:read-type-policy} returns a default value for the \texttt{firstName} of the \texttt{User} type when the cache has no value stored. If the cache contains a value for that field, the value is returned unmodified.

\ifshowListings
\begin{listing}[H]
    \begin{minted}{typescript}
new InMemoryCache({
  typePolicies: { User: { fields: {
    firstName: {
      read(value, options) {
        return value ?? 'Unknown';
      }
    }
  }}}
});
    \end{minted}
    \caption{Provide a default value for the \texttt{firstName} field.}\label{code:background:graphql:read-type-policy}
\end{listing}
\fi

\paragraph{Merge function}

The \texttt{merge} function for a field is called whenever the field is about to be written with an incoming value. The field's new value inside the cache is set to the \texttt{merge} function's return value instead of the original value. An everyday use case for the \texttt{merge} function is to define a field policy for a field that holds an array. By default, the existing collection is completely replaced by the incoming array. However, sometimes, both arrays should be concatenated. This pattern is frequently used with paginated lists, where the incoming page should be merged with the existing pages. The parameters of the function are the current value, the incoming value, and the options, just like in the \texttt{read} function. \cite{misc:-:background:graphql:apollo-client-cache-reading-writing}

\bigskip

\noindent Listing \ref{code:background:graphql:write-type-policy} shows the usage of a \texttt{merge} function. Whenever a new transaction for the bank account is fetched, the existing and the incoming transactions are merged. Initially, \texttt{existing} is undefined; therefore, \texttt{existing} is assigned a default parameter.

\ifshowListings
\begin{listing}[H]
    \begin{minted}{typescript}
new InMemoryCache({
  typePolicies: { BankAccount: { fields: {
    transaction: {
      merge(existing = [], incoming: unknown[], options) {
        return [...existing, ...incoming];
      },
    },
  }}},
});
    \end{minted}
    \caption{Merge the existing and incoming transactions in a \texttt{merge} function.}\label{code:background:graphql:write-type-policy}
\end{listing}
\fi
