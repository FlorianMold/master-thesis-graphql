\section{Micro-Frontend Architecture}

Micro-frontends should bring the same advantages of microservices from the backend to the frontend. Instead of creating a large frontend monolith, a micro-frontend architecture contains many small applications. The advantage is that every micro-frontend can be developed and deployed by a separate team. \cite{book:2020:geers:micro-frontends-in-action} The difference between frontend-monoliths and micro-frontends can be seen in figure \ref{figure:state-of-the-art:ui-monolith-micro-frontend}.

\ifshowImages
\begin{figure}[H]
\centering
\includegraphics[width=0.8\linewidth]{images/ui-monolith-micro-frontends.jpeg}
\caption{A comparison between frontend-monoliths and micro-frontends.}\label{figure:state-of-the-art:ui-monolith-micro-frontend}
\end{figure}
\fi

Benefits gained from working with microservices on the backend are lost, when working with a monolithical frontend. With a monolithic frontend, the ability to deploy independently is lost. The entire frontend has to be deployed at once. Another problem is, that distinct operations are not really possible. If one part of the frontend is broken, there is a good chance that the entire frontend is broken. Another problem is the parallel development. The speed of development cannot be increased because it is very difficult to have multiple teams working on one frontend application. \cite{misc:2019:leitner:micro-frontends}

The term micro-frontend can be misleading, as can the term microservice. It has no meaning in terms of the size of the application. It can be a simple widget that only displays data, or a full-blown one-page application. Ideally, a micro frontend covers an area of the entire frontend application.


Micro-frontends try to apply the same principles from the microservice architecture to frontend development. Often times a microservice architecture with is developed by several teams has only one frontend application. Therefore, when adding new features a single team can be overwhelmed. Like a microservice architecture, a micro-frontend architecture focuses on developing many small frontend-applications, instead of developing a large software monolith. Each micro-frontend can be developed independently by another team. But a challenge is that the micro-frontend should appear as a single application to the user. Therefore, the different applications have to be integrated, which can be a challenge.

The term micro-frontend should not lead to false conclusions about the size of an application. The size of micro-frontends can vary. It can range from a simple login to a complex single-page application.

Building micro-frontends with the web allows different strategies of integrating the applications. Three different strategies exist to combine multiple micro-frontends into an app-shell. The client-side integration, the server-side integration and the combination of these two strategies and the combination of both strategies.

\subsection{Generic APIs vs Consumer Driven APIs}

The big decision in micro-frontend API development is to use either generic or consumer-oriented APIs. The difference is that generic APIs place great emphasis on reusability, while consumer-oriented APIs tailor the APIs to the customer.

\subsubsection{Generic APIs}

Generic APIs refer to APIs that are very general and can be used by different clients. However, this type of API has two major drawbacks. Over-fetching describes the problem of getting more data than is needed. Over-requesting describes the problem of needing multiple requests to get the data for a use case. Both problems are discussed in more detail in the next paragraphs. \cite{misc:2019:leitner:backend-for-frontends}

\paragraph{Over-Fetching}

For example, a contact service provides a contact-model that includes customer-number, first-name, second-name, uid-number and the address of the user, as seen in listing \ref{code:state-art:over-fetching}. However, one requirement of the application is to display only a contact's first and last name inside the header. Only two fields of the model are used, and the rest are unnecessarily queried. \cite{misc:2019:leitner:backend-for-frontends}

\ifshowListings
\begin{listing}[H]
\begin{minted}{typescript}
interface ContactModel {
  id: string;
  customerNumber: string;
  firstName: string;
  secondName: string;
  uidNumber: string;

  Address: {
    id: string;
    postalCode: string;
    location: string;
    Country: string;
  }
}
\end{minted}
\caption{Contact-Model that contains too much fields for the requirement.}\label{code:state-art:over-fetching}
\end{listing}
\fi

\paragraph{Over-Requesting}

Attempting to solve the problem of over-fetching by reducing the amount of data set that is returned leads directly to this problem. Listing \ref{code:state-art:over-requesting} shows the problem of over-requesting. If another requirement inside the application should display the address alongside the contact, two requests have to be performed every time. Afterwards, the two data sets have to be merged, which leads to high complexity on the client side. \cite{misc:2019:leitner:backend-for-frontends}

\ifshowListings
\begin{listing}[H]
\begin{minted}{typescript}
interface ContactModel {
  id: string;
  customerNumber: string;
  firstName: string;
  secondName: string;
  uidNumber: string;

  address_id: string;
}
\end{minted}
\caption{Contact-Model model that links the address-model with an id.}\label{code:state-art:over-requesting}
\end{listing}
\fi

\subsubsection{Consumer Driven APIs}

Consumer-driven APIs are the opposite of generic APIs. They follow the idea of providing the client with exactly the data it needs. Following the example above, the contact service would have an endpoint that returns only the first and last name as required for the request. These endpoints make communication with a client very simple and there is not the problem of over-fetching and over-requesting. However, creating an endpoint for each request creates an unmanageable set of endpoints. \cite{misc:2019:leitner:backend-for-frontends}


To solve these problems, the backend-for-frontend pattern is often used. This pattern provides each client with its own API, which specialized for the needs of the client. \cite{book:2018:richardson:microservices-patterns}

\ifshowImages
\begin{figure}[H]
\centering
\includegraphics[width=0.8\linewidth]{images/ui-bff-architecture.jpeg}
\caption{Frontend architecture with the backend-for-frontend pattern.}\label{figure:state-of-the-art:ui-bff-architecture}
\end{figure}
\fi

Figure \ref{figure:state-of-the-art:ui-bff-architecture} shows an exemplary micro-frontend architecture using the backend-for-frontend pattern. Each frontend has a service that retrieves data only for that specific client. Because the backend-for-frontends function as gateway to the domain services, the domain services can stay very generic and be reused by different clients. Backend-for-frontends should implement only the presentation logic that puts the data into the form that the client needs. It should avoid storing state. \cite{misc:2019:leitner:backend-for-frontends}

With this architectural approach the backend-for-frontend and the frontend form a single deployment unit. If one application is changed, the other needs needs to adapt the changes. GraphQL is a perfect technology for implementing a backend-for-frontend, because it is specifically designed for implementing the presentation-layer.

\subsection{Characteristics}

Micro-frontends tend to follow the same characteristics as microservices.

\subsubsection{Autonomous}

Technically a micro-frontend is a completely independent and runnable application.
The integration of the micro-frontends happens only through the frontend. The different micro-frontends are composed withing an app-shell. The application shell is a separate application that is usually the entry-point for the user to interact with all micro-frontends. The app-shell also provides the layout of the page and defines where the micro-frontends are placed. \cite{book:2020:geers:micro-frontends-in-action}

\subsubsection{Technology Agnostic}

Just as microservices architectures, micro-frontend architectures can be technology agnostic. The current frontend development landscape offers a lot of JavaScript frameworks to choose from.

\subsubsection{Independently Depoyable}

The autonomy of micro-frontends offer the possibility for independent deployments. A large monolithical micro-frontends is trickier to deploy. There is no need to have communication over multiple teams to deploy the application.


\subsubsection{Small and Easy to Maintain}

Because micro-frontends only cover a small domain of an application, the source code is smaller and easier to understand. A smaller codebase is especially helpful for understanding a program. 
Due to the easier understanding of the domain, the application can be easier rewritten with a state of the art technology. \cite{book:2020:geers:micro-frontends-in-action}

\subsubsection{Resilience}

Micro-frontends offer the possibility to build an application by composing multiple small standalone application into a fully fledged application. Depending on the integration strategy micro-frontends are usually combined at runtime. 
% \cite{motivation}

A micro-frontend architecture provide better failure isolation. One micro-frontend crashing does not have an effect on the other micro-frontends inside the application. Some parts of the application might not work, but other parts of the application are still useable. The app-shell can react to a failure and tell users that the application is not working as expected and will be available back soon. For

\subsection{Downsides}

Due to the many advantages of micro-frontends there are also a downsides using this architectural approach. The independent development comes with the disadvantage of having redundancies. Each micro-frontend needs a separate build-process and also a continuous integration pipeline.
\cite{book:2020:geers:micro-frontends-in-action}


\subsection{Integration strategies}

Micro-frontends can be integrated in different ways. The integration strategy depends on the requirements of the system. They can be composed using a client-side integration strategy, a server-side strategy or a combination of both strategies.

\subsubsection{Server-Side Integration}

Server-side composition is usually done by a service that sits between the client and the backend. \cite[60]{book:2020:geers:background:micro-frontends:micro-frontends-in-action} The server responds with the references to micro-frontends that should be included, as well as their required assets. The service in the middle intercepts that response and replaces the references to the micro-frontends with the real content, before the response is sent to the browser. The micro-frontends are included in their position, where they later appear in the HTML. An example include can be seen in listing \ref{listing:background:micro-frontends:server-side-include}. The other micro-frontends are referenced with URL's. \cite[61-63]{book:2020:geers:background:micro-frontends:micro-frontends-in-action}

\ifshowListings
\begin{listing}[H]
    \begin{minted}{html}
<html>
  <body>
    <!--#include virtual="/erp/dashboard" -->
  </body>
</html>
    \end{minted}
    \caption{An example server-side include.}\label{listing:background:micro-frontends:server-side-include}
\end{listing}
\fi

\bigskip

\noindent One advantage of server-side integration is the fast first load performance, which the principle of progressive enhancement. \cite{book:2010:parker:background:micro-frontends:designing-with-progressive-enhancement} The browser just fetches the HTML and renders it. It does not have to assemble parts of a page, like with client-side integration. The computation is only done on the server, which reduces the strain on the users device. \cite{book:2020:geers:background:micro-frontends:micro-frontends-in-action} But assets like stylesheets and images still have to be fetched from the server. Server Side Integration is useful, if the main concern of the application is to present static content to the end user. Instant reaction to the inputs of the users is not needed. \cite[83]{book:2020:geers:background:micro-frontends:micro-frontends-in-action}

\subsubsection{Client-Side Integration}

When the application should react promptly to user input, a client-side integration strategy is preferred. For example, when developing an online marketplace, the user should be able to add items to the cart without making a complete roundtrip to the server to see the updated value. The application should provide a seamless user experience, as the end user just uses one application. Modern Frameworks like Angular, React offer the development of Single Page Applications, which provide reactive, client-side rendered applications. The HTML Markup is produced on the client, instead of the server. \cite{book:2020:geers:background:micro-frontends:micro-frontends-in-action}

\bigskip

\noindent Client-side integration can be achieved through different approaches. The simplest approach is to combine the micro-frontends by linking the different applications with Hyperlinks together. Each micro-frontend is deployed and accessible via a different URL. The different micro-frontend applications are then linked together with Hyperlinks. As the approach implies the switch to another micro-frontend requires a complete page reload and a roundtrip to the server. But the integration with Hyperlinks breaks the SPA approach. This strategy makes it necessary that every micro-frontend is accessible via it's own URL and that it can be served as a standalone application.

\bigskip

\noindent Another client-side approach is to combine micro-frontend with iFrames or Web-Components. Integrating micro-frontends with this approach enables the page to still be a SPA. The client can navigate multiple different micro-frontends without even noticing and no page-reload is needed. iFrames are an isolated are if the website with their own browser context \cite[35]{book:2020:geers:background:micro-frontends:micro-frontends-in-action}, Web Components are self-created HTML elements that are embedded into the DOM of the browser \cite[103]{book:2019:farrell:background:micro-frontends:web-components-in-action}. Integrating applications with the client-side strategy using Module Federation is explained in more detail in a following section.


\input{chapters/background/micro-frontend/communication.tex}

\subsection{Backend-For-Frontend Pattern}

Every microservice provides its functionality to consumers with APIs. But it is not advisable that clients directly communicate with microservice APIs. Microservice offer fine-grained interfaces which were made especially for the communication between microservices. Therefore, the client usually has to make multiple requests to fetch the data needed for a view. ([7] S. Newman, Building microservices: designing fine-grained systems, First Edition. Beijing Sebastopol, CA: O’Reilly Media, 2015, (ISBN 978-1-4919-5035-7).

This leads to many requests, which is also known as over-requesting.

Another problem could be that a cluster of microservices use another form of communication. For example an asynchronous message-bus or another protocol like GRPC. There is the next problem. Clients usually communicate using synchronous communication, where microservices could use asynchronous communication. Without an adapter in between, the communication will not work properly. Even if the communication is possible, the client needs to know many details (IP-address) about the cluster of microservices. And the client might have to connect to multiple microservice to fetch the data needed to display one view. Therefore the client has to join the data in-memory. Changing the API of a microservice would have a ripple effect on the requests on the frontends, because they would have to be changed in many places.

[1] C. Richardson, Microservices patterns: with examples in Java. Shelter Island, New York: Manning Publications, 2019, (ISBN 978-1-61729-454-9).

To solve this problem the clients communicate with an API gateway or a more client centric backend-for-frontend service. Internally these services communicate with the microservice-cluster. An API gateway is a service that represents an abstraction of the microservice APIs and is an entry point to the microservice-cluster. The main task of a gateway is to forward tasks to the correct microservice. The even might implement functionalities like authorization and authentication or transform the protocol. Like transforming HTTP to GRPC. With API gateways it is also easier to split a microservice into two for example, without a ripple effect to change all clients as well.

But the problem with API gateways is the ownership. Multiple teams will add their functionality to the gateway and might come into conflict. The APIs are often not suited for the needs of clients and it has to be avoided that client logic is developed into the API gateway.

