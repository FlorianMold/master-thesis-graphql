\subsection{Backend-For-Frontend Pattern}

Every microservice provides its functionality to consumers with APIs. But it is not advisable that clients directly communicate with microservice APIs. Microservice offer fine-grained interfaces which were made especially for the communication between microservices. Therefore, the client usually has to make multiple requests to fetch the data needed for a view. ([7] S. Newman, Building microservices: designing fine-grained systems, First Edition. Beijing Sebastopol, CA: O’Reilly Media, 2015, (ISBN 978-1-4919-5035-7).

This leads to many requests, which is also known as over-requesting.

Another problem could be that a cluster of microservices use another form of communication. For example an asynchronous message-bus or another protocol like GRPC. There is the next problem. Clients usually communicate using synchronous communication, where microservices could use asynchronous communication. Without an adapter in between, the communication will not work properly. Even if the communication is possible, the client needs to know many details (IP-address) about the cluster of microservices. And the client might have to connect to multiple microservice to fetch the data needed to display one view. Therefore the client has to join the data in-memory. Changing the API of a microservice would have a ripple effect on the requests on the frontends, because they would have to be changed in many places.

[1] C. Richardson, Microservices patterns: with examples in Java. Shelter Island, New York: Manning Publications, 2019, (ISBN 978-1-61729-454-9).

To solve this problem the clients communicate with an API gateway or a more client centric backend-for-frontend service. Internally these services communicate with the microservice-cluster. An API gateway is a service that represents an abstraction of the microservice APIs and is an entry point to the microservice-cluster. The main task of a gateway is to forward tasks to the correct microservice. The even might implement functionalities like authorization and authentication or transform the protocol. Like transforming HTTP to GRPC. With API gateways it is also easier to split a microservice into two for example, without a ripple effect to change all clients as well.

But the problem with API gateways is the ownership. Multiple teams will add their functionality to the gateway and might come into conflict. The APIs are often not suited for the needs of clients and it has to be avoided that client logic is developed into the API gateway.
