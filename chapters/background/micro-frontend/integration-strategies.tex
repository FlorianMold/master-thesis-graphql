\subsection{Integration strategies}\label{subsection:background:micro-frontend-architecture:integration-strategies}

Micro-frontends can be integrated with different strategies. The integration strategy depends on the requirements of the system. They can be composed using a client-side integration strategy, a server-side strategy, or a combination of both.

\subsubsection{Server-Side Integration}\label{subsubsection:background:micro-frontend-architecture:integration-strategies:server-side-integration}

A Service between the client and the server usually does server-side composition. \cite[60]{book:2020:geers:background:micro-frontends:micro-frontends-in-action} The server responds with references to micro-frontends that should be included and their required assets. The service in the middle intercepts that response and replaces the references to the micro-frontends with the actual content before the response is sent to the browser. The micro-frontends are included in their position and later appear in the \ac{HTML}. An example of a server-side include can be seen in listing \ref{code:background:micro-frontends:server-side-include}. The other micro-frontends are referenced with \acp{URL}. \cite[61-63]{book:2020:geers:background:micro-frontends:micro-frontends-in-action}

\ifshowListings
\begin{listing}[H]
    \begin{minted}{html}
<html>
  <body>
    <!--#include virtual="/erp/dashboard" -->
  </body>
</html>
    \end{minted}
    \caption{An example for a server-side include.}\label{code:background:micro-frontends:server-side-include}
\end{listing}
\fi

\bigskip

\noindent One advantage of server-side integration is the fast first-load performance, which is the principle of progressive enhancement. \cite{book:2010:parker:background:micro-frontends:designing-with-progressive-enhancement} The browser fetches the \ac{HTML} and renders it. It does not have to assemble parts of a page, like with client-side integration. The computation is only done on the server, which reduces the strain on the user's device. \cite{book:2020:geers:background:micro-frontends:micro-frontends-in-action} However, assets like stylesheets and images must still be fetched from the server. Server Side Integration is helpful if the application's primary concern is presenting static content to the end user and instant reaction to the users' inputs is unnecessary.  \cite[83]{book:2020:geers:background:micro-frontends:micro-frontends-in-action}

\subsubsection{Client-Side Integration}\label{subsubsection:background:micro-frontend-architecture:integration-strategies:client-side-integration}

When the application should react promptly to user input, a client-side integration strategy is preferred. For example, when developing an online marketplace, the user should be able to add items to the cart without making a complete roundtrip to the server to see the updated value. The application should provide a seamless user experience, as the end user just uses one application. Modern Frameworks like Angular and React offer the development of Single Page Applications, which provide reactive, client-side rendered applications. The \ac{HTML} Markup is produced on the client instead of the server. \cite{book:2020:geers:background:micro-frontends:micro-frontends-in-action}

\bigskip

\noindent Client-side integration can be achieved through different approaches. The most straightforward approach combines the micro-frontends by linking the different applications with Hyperlinks. Each micro-frontend is deployed and accessible via a different \ac{URL}, and the different micro-frontend applications are then linked with Hyperlinks. As the approach implies, switching to another micro-frontend requires a complete page reload and a roundtrip to the server. However, the integration with Hyperlinks breaks the \ac{SPA} approach. This strategy makes it necessary that every micro-frontend is accessible via its \ac{URL} and that it can be served as a standalone application.

\bigskip

\noindent Another client-side approach is to combine micro-frontend with iFrames or Web-Components. Integrating micro-frontends with this approach enables the page to be a \ac{SPA} still. The client can navigate multiple micro-frontends without noticing, and no page reloads are needed. An iFrame is an isolated area with its own browser context \cite[35]{book:2020:geers:background:micro-frontends:micro-frontends-in-action}, Web Components are self-created \ac{HTML} elements that are embedded into the DOM of the browser \cite[103]{book:2019:farrell:background:micro-frontends:web-components-in-action}. Integrating applications with the client-side strategy using Module Federation is explained in more detail in Section \ref{subsubsection:background:micro-frontend:module-federation:101}.
