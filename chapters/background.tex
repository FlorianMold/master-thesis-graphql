\chapter{Background}

This chapter will give insights into the most important topics and technologies relevant for this thesis. At the beginning, the 

Afterwards a brief overview of GraphQL is presented. Additional

\section{Micro-Frontend Architecture}

\subsection{Characteristics}

\subsection{Integration strategies}

\subsection{Communication}

\subsection{Backend-For-Frontend Pattern}

\section{GraphQL}

The officical documentation states that GraphQL is a query language for APIs and a runtime for fulfilling those queries with your existing data. GraphQL provides a complete and understandable description of the data in your API, gives clients the power to ask for exactly what they need and nothing more, makes it easier to evolve APIs over time, and enables powerful developer tools. GraphQL is implemented in many languages and frameworks. It has a large ecosystem of libraries and tools.

\subsection{Origins and history}

GraphQL was initially developed at Facebook in 2012. In 2015 the project was made open-source and available to the public. The motivation behind the initial development of GraphQL was the limited flexibility of available API technologies like REST.

\subsection{Apollo Server and Client}

GraphQL is a query language and follows specific rules. The development of a GraphQL server and client is up to the application developer. Facebook has developed their own implementation of a GraphQL server and client. The server is called GraphQL.js and the client is called Relay.

Apollo 

Apollo Client is m

\subsubsection{How does the cache work?}

This section describes how the cache of apollo works.

By default all GraphQL requests made with \textbf{ApolloClient} are cached inside the browsers memory. This is needed, to ensure to reduce round-trips to the server in subsequent requests of the same query, because the requested data can be served from the cache.

The caching mechanism reduces the load of the server, but it introduces issues with cache management.