\chapter{Results}\label{chapter:results}

This chapter measures whether a micro-frontend architecture with GraphQL and a shared caching layer can provide a performance improvement over a separated cache. In total, the micro-frontend architecture implements four \acp{SPA} and nine widgets. The major part of the implementation was done using Angular, but one single widget was implemented in React. This was done to showcase whether the shared caching layer could be used with every technology. Furthermore, a \ac{BFF} service was developed in GraphQL that is tailored to the needs of the micro-frontends. The \ac{BFF} service is used to aggregate the data from the microservices and to provide it to the micro-frontends. An overview of the prototypical architectures communicates with the GraphQL \ac{API} is shown in listing \ref{fig:results:micro-frontend-prototype}.

\ifshowImages
\begin{figure}[H]
  \centering
  \includegraphics[width=0.8\linewidth]{images/results/micro-frontend-prototype.png}
  \caption{Architecture of the micro-frontend prototype.}\label{fig:results:micro-frontend-prototype}
\end{figure}
\fi

\section{Performance measurement}\label{section:results:performance-measurement}

This section explains how the micro-frontend architecture was evaluated in terms of the hypothesis. Three distinct approaches were identified to measure the performance of the shared GraphQL caching layer. The architecture allows switching easily between these three approaches.

\begin{enumerate}
  \item \textbf{Separate Cache and no reduced queries}: All remote modules use a separate cache and no queries are reduced with the help of the cache.
  \item \textbf{Shared Cache and no reduced queries}: The remote modules share the same cache instance and no queries are reduced with the help of the cache.
  \item \textbf{Shared Cache and reduced queries}: ALL remote modules share the same instance of the cache and queries are reduced by utilizing the cache.
\end{enumerate}

\noindent To measure and compare the performance of these three approaches, two exemplary paths through the application were planned. These paths were intended to show how many network requests were made to the GraphQL \ac{API} and how much network traffic was generated in the process. To make the measurement as close as possible to a real application, a large amount of mock data was generated for the GraphQL \ac{API}. With this large amount of data, it is easier to measure the differences in response size. Smaller datasets only make a big difference, if the application is used for longer periods. The next section details the results of the first path through the application.

\subsection{Evaluation}\label{subsection:results:performance-measurement:evaluation}

This section describes how the shared caching layer and the reduction of queries were tested for the prototypical micro-frontend architecture. It explains the user journey through the application and shows the results for the three different approaches. The figure \ref{fig:results:evaluation-first-path} shows the steps through the micro-frontend architecture that were used to measure the possible performance improvements of the shared caching layer and the reduction of queries. The client has to perform 13 steps throughout the application, which involves almost every available GraphQL query. The dashoard yields some problems, when starting the evaluation there. All widgets are created and start to fetch their data at the same time. Therefore, it can easily happen that multiple widgets fetch the same queries from the GraphQL \ac{API} because the data is not already in the cache. This is a problem that is difficult to circumvent and leads to a lot of theoretically unnecessary network requests. The evaluation is performed with an unauthenticated user.

\ifshowImages
\begin{figure}[H]
\centering
\includegraphics[width=1\linewidth]{images/results/evaluation-first-path.png}
\caption{A user journey through the application to measure the performance of the micro-frontend architecture.}\label{fig:results:evaluation-first-path}
\end{figure}
\fi

\noindent Without the use of a caching system, the GraphQL \ac{API} would have to run 59 queries to provide the data for the path through the application. How the prototypical micro-frontend architecture can be configured to use one of the three approaches is already explained in section \ref{section:applied-methods:shared-caching-layer} and section \ref{subsection:applied-methods:query-reduction:testing-query-reduction}. The following sections describe and compare the results of the approaches in more detail.

\subsubsection{Separate Cache and no reduced queries}\label{subsubsection:results:performance-measurement:separate-cache-no-reduction}

In this approach, each micro frontend has a separate instance of the GraphQL client and \texttt{InMemoryCache}. The queries are not reduced using the cache and the custom implementation. After the client completes the journey through the application, the following metrics were collected.

\begin{itemize}
  \item 47 network requests to the GraphQL \ac{API}
  \item 10.78MB transferred
\end{itemize}

\noindent The \texttt{GRAPHQL\_CLIENT\_OPTIONS\_CONFIG} and \texttt{REDUCE\_QUERY\_OPTIONS} injection tokens have to be configured the following way A more detailed description of the configuration options can be found in section \ref{section:applied-methods:shared-caching-layer} and in section \ref{subsection:applied-methods:query-reduction:testing-query-reduction}:

\begin{itemize}
  \item \texttt{shareCache: false}
  \item \texttt{reduceQueries: false}
\end{itemize}

\noindent 47 network requests have to be made to the GraphQL backend which can be seen in figure \ref{fig:results:no-shared-cache-no-reduction}. The figure shows 53 requests in total, but six requests have to be subtracted because they are needed to make the prototypical architecture work. They load the micro-frontends from their remote locations and fetch their settings. These requests are only needed for the functionality of the micro-service architecture.

\ifshowImages
\begin{figure}[H]
\centering
\includegraphics[width=0.6\linewidth]{images/results/1-attempt/no-shared-cache-no-reduction.png}
\caption{All requests made during the measurement of the first approach.}\label{fig:results:no-shared-cache-no-reduction}
\end{figure}
\fi

\noindent The total size of the requests was 17.46 KB and the size of the responses was 10.78 MB. The 47 queries retrieve a total of 81510 records from the GraphQL backend.

\subsubsection{Shared Cache and no reduced queries}\label{subsubsection:results:performance-measurement:shared-cache-no-reduction}

In this approach, an instance of the cache is shared by all micro-frontends, but the GraphQL queries are not reduced with data already present in the \texttt{InMemoryCache}. After the client completes the journey through the application, the following metrics were collected:

\begin{itemize}
  \item 36 network requests to the GraphQL \ac{API}
  \item 8.5 MB transferred
\end{itemize}

\noindent The \texttt{GRAPHQL\_CLIENT\_OPTIONS\_CONFIG} and \texttt{REDUCE\_QUERY\_OPTIONS} injection tokens have to be configured the following way:

\begin{itemize}
  \item \texttt{shareCache: true}
  \item \texttt{reduceQueries: false}
\end{itemize}

\noindent 36 requests have to be made to the GraphQL backend which can be seen in figure \ref{fig:results:no-shared-cache-no-reduction}. Six requests have to be deducted (\texttt{settings.json}, \texttt{module-federation.manifest.json}, \dots) like in the previous section.

\ifshowImages
\begin{figure}[H]
\centering
\includegraphics[width=0.6\linewidth]{images/results/1-attempt/shared-not-reduced-cache.png}
\caption{All requests made during the measurement of the second approach.}\label{fig:results:shared-cache-no-reduction}
\end{figure}
\fi

\noindent The total size of the queries was 15.176 KB and the size of the responses was 8.5 MB. The 36 queries retrieve a total of 51319 records from the GraphQL backend.

\subsubsection{Shared cache, query reduction}\label{subsubsection:results:performance-measurement:separate-cache-reduction}

With this approach, the same instance of the cache is shared between all of the micro-frontends in the architecture, and the queries are reduced with already existing data inside the cache. After the client completes the journey through the application, the following metrics were collected.

\begin{itemize}
  \item 36 network requests to the GraphQL \ac{API}
  \item 8.4 MB transferred
\end{itemize}

\noindent The \texttt{GRAPHQL\_CLIENT\_OPTIONS\_CONFIG} and \texttt{REDUCE\_QUERY\_OPTIONS} injection tokens have to be configured the following way:

\begin{itemize}
  \item \texttt{shareCache: true}
  \item \texttt{reduceQueries: true}
\end{itemize}

\noindent 36 requests have to be made to the GraphQL backend which can be seen in figure \ref{fig:results:no-shared-cache-no-reduction}. Six requests have to be deducted (\texttt{settings.json}, \texttt{module-federation.manifest.json}, \dots) like in the previous sections.

\ifshowImages
\begin{figure}[H]
\centering
\includegraphics[width=0.6\linewidth]{images/results/1-attempt/no-shared-cache-no-reduction.png}
\caption{All requests made during the measurement of the third approach.}\label{fig:results:shared-cache-reduction}
\end{figure}
\fi

\noindent The total size of the queries was 13.533 KB and the size of the responses was 8.37 MB. The 36 queries retrieve a total of 51319 records from the GraphQL backend.

\section{Compare the results of the first user journey}\label{section:results:comparison-first-journey}

This section explains the first user journey through the prototype and shows the three approaches' results. Figure \ref{fig:results:evaluation-first-path} shows the steps through the micro-frontend prototype used to measure the possible performance improvements of the shared caching layer and the reduction of queries. The client has to perform 13 steps throughout the prototype, which involves fetching almost every available GraphQL query. The dashboard micro-frontend yields some problems if the evaluation is started there. All widgets start to fetch their data simultaneously. Therefore, it can easily happen that multiple widgets fetch the same queries from the GraphQL \ac{API} because the data is not in the cache yet. This problem leads to a lot of theoretically unnecessary network requests and is difficult to circumvent. The evaluation shown in the figure is performed with an unauthenticated user. This section takes the measurements and compares the different approaches regarding request size, response size, the number of requests, and the total records fetched.

\ifshowImages
\begin{figure}[H]
  \centering
  \includegraphics[width=1\linewidth]{images/results/evaluation-first-path.png}
  \caption{A user journey through the application to measure the performance of the micro-frontend architecture.}\label{fig:results:evaluation-first-path}
\end{figure}
\fi

\noindent Without a caching system, the GraphQL \ac{API} would have to execute 59 queries to provide the data for the journey through the prototype. How to configure the prototype to use one of the three approaches has already been discussed in Section \ref{subsection:applied-methods:shared-caching-layer:graphql-client-creationsubsection:applied-methods:shared-caching-layer:graphql-client-creation} and Section \ref{subsection:applied-methods:query-reduction:testing-query-reduction}. The following sections describe and compare the results in more detail.

\subsubsection{Separate Cache and no reduced queries}\label{subsubsection:results:performance-measurement:separate-cache-no-reduction}

In this approach, each micro-frontend has its own instance of the Apollo Client and \texttt{InMemoryCache}. The queries were sent unaltered to the GraphQL \ac{API}. The following metrics were collected after the user journey was complete:

\begin{itemize}
  \item 47 network requests to the GraphQL \ac{API}
  \item 10.80 MB transferred (request size + response size)
\end{itemize}

\noindent The \texttt{GRAPHQL\_CLIENT\_OPTIONS\_CONFIG} and \texttt{REDUCE\_QUERY\_OPTIONS} injection tokens have to be configured the following way:

\begin{itemize}
  \item \texttt{shareCache: false}
  \item \texttt{reduceQueries: false}
\end{itemize}

\noindent A more detailed description of the configuration options can be found in Section \ref{subsection:applied-methods:shared-caching-layer:graphql-client-creation} and in Section \ref{subsection:applied-methods:query-reduction:testing-query-reduction}. 47 network requests have to be sent to the GraphQL \ac{API}, which can be seen in Figure \ref{fig:results:no-shared-cache-no-reduction}. The figure shows 54 requests, but 7 requests have to be subtracted because they are needed to make the integration of micro-frontends work. These requests fetch the micro-frontends and their settings from the remote location.

\ifshowImages
\begin{figure}[H]
  \centering
  \includegraphics[width=0.8\linewidth]{images/results/1-attempt/no-shared-cache-no-reduction.jpg}
  \caption{Requests made during the measurement of the first approach.}\label{fig:results:no-shared-cache-no-reduction}
\end{figure}
\fi

\noindent The total size of the requests was 17.46 KB, and the responses were 10.78 MB. The 47 queries retrieve a total of 81510 records from the GraphQL  \ac{API}.

\subsubsection{Shared Cache and no reduced queries}\label{subsubsection:results:performance-measurement:shared-cache-no-reduction}

In this approach, a single cache instance is shared by all micro-frontends. The queries were sent unaltered to the GraphQL \ac{API}. The following metrics were collected after the user journey was complete:

\begin{itemize}
  \item 36 network requests to the GraphQL \ac{API}
  \item 8.45 MB transferred (request size + response size)
\end{itemize}

\noindent The \texttt{GRAPHQL\_CLIENT\_OPTIONS\_CONFIG} and \texttt{REDUCE\_QUERY\_OPTIONS} injection tokens have to be configured the following way:

\begin{itemize}
  \item \texttt{shareCache: true}
  \item \texttt{reduceQueries: false}
\end{itemize}

\noindent 36 requests have to be sent to the GraphQL \ac{API}, which can be seen in Figure \ref{fig:results:shared-cache-no-reduction}. Seven requests have to be deducted (\texttt{settings.json}, \texttt{module-federation.manifest.json}, \dots) as in the previous section.

\ifshowImages
\begin{figure}[H]
  \centering
  \includegraphics[width=0.8\linewidth]{images/results/1-attempt/shared-not-reduced-cache.jpg}
  \caption{Requests made during the measurement of the second approach.}\label{fig:results:shared-cache-no-reduction}
\end{figure}
\fi

\noindent The total size of the queries was 15.18 KB, and the size of the responses was 8.43 MB. The 36 queries retrieve a total of 51319 records from the GraphQL \ac{API}.

\subsubsection{Shared cache, query reduction}\label{subsubsection:results:performance-measurement:separate-cache-reduction}

In this approach, a single cache instance is shared between all micro-frontends. The queries are reduced by removing fields that are already inside the cache. The following metrics were collected after the user journey was complete:

\begin{itemize}
  \item 36 network requests to the GraphQL \ac{API}
  \item 8.39 MB transferred (request size + response size)
\end{itemize}

\noindent The \texttt{GRAPHQL\_CLIENT\_OPTIONS\_CONFIG} and \texttt{REDUCE\_QUERY\_OPTIONS} injection tokens have to be configured the following way:

\begin{itemize}
  \item \texttt{shareCache: true}
  \item \texttt{reduceQueries: true}
\end{itemize}

\noindent 36 requests have to be made to the GraphQL \ac{API}, which can be seen in Figure \ref{fig:results:shared-cache-reduction}. Seven requests have to be deducted (\texttt{settings.json}, \texttt{module-federation.manifest.json}, \dots) as in the previous sections.

\ifshowImages
\begin{figure}[H]
  \centering
  \includegraphics[width=0.8\linewidth]{images/results/1-attempt/shared-reduced-cache.jpg}
  \caption{Requests made during the measurement of the third approach.}\label{fig:results:shared-cache-reduction}
\end{figure}
\fi

\noindent The total size of the queries was 13.53 KB, and the size of the responses was 8.37 MB. The 36 queries retrieve a total of 51319 records from the GraphQL \ac{API}.

\subsection{Compare the first- and second-approach}\label{subsection:results:comparison-first-second-approach}

When comparing the first approach to the second, there is a significant difference in the number of network requests to the GraphQL \ac{API} and the size of the requests and responses, as seen in Table \ref{table:results:size-comparison-first-path-no-cache-no-reduction-cache-no-reduction}. The second approach requires 11 fewer network requests than the first approach. Since the queries are not altered for this comparison, the additional network requests are responsible for the overall difference in request- and response size. The 11 additional requests from the first approach send an additional 2.29 KB to the \ac{API} and return about an additional 2.34 MB from the \ac{API}. Therefore, 22\% of the total response size can be saved using a shared caching layer for all micro-frontends. Another interesting observation is that the shared cache approach retrieves 30191 fewer records than the naive approach, about 37\% of the total records returned. Many queries need to be retrieved more than once in the first approach, hence the large difference in the number of records.

\ifshowTables
\begin{table}[H]
  \begin{tabular}{|l|l|l|l|l|}
  \hline
    & \textbf{Req. Size (B)} & \textbf{Resp. Size (B)} & \textbf{Requests} & \textbf{Records} \\
    \hline
    \textbf{No Reduction, Separate Cache} & 17462 & 10780656 & 47 & 81510 \\
    \hline
    \textbf{No Reduction, Shared Cache} & 15176 & 8437211 & 36 & 51319 \\
    \hline
    \hline
    \textbf{Diff (B)} & \textbf{2286} & \textbf{2343445} & \textbf{11} & \textbf{30191} \\
    \hline
    \textbf{Reduction (\%)} & \textbf{13\%} & \textbf{22\%} & \textbf{23\%} & \textbf{37\%} \\
    \hline
  \end{tabular}
  \caption{First Journey: Compare the requests and responses of the first- and second-approach.}\label{table:results:size-comparison-first-path-no-cache-no-reduction-cache-no-reduction}
\end{table}
\fi

\noindent The following enumeration shows which and how often a GraphQL query was discarded when using a shared caching layer between the micro-frontends compared to a separate cache:

\begin{itemize}
  \item allCountries: 2
  \item allSalutations: 2
  \item allTitles: 2
  \item allArticleUnits: 1
  \item allCurrencies: 1
  \item allVats: 1
  \item allSalesCountries: 1
  \item allInvoiceTypes: 1
\end{itemize}

\noindent The data from the omitted requests is typically used for populate selection controls within detail views and has to be retrieved repeatedly in each micro-frontend. The first three queries are used for widgets on the dashboard, the Contact application, and the User application. The last five queries are used for Dashboard widgets and the Sales application.

\subsection{Compare the first- and third-approach}\label{subsection:results:comparison-first-third-approach}

As in the previous comparison, there is the same difference in the number of network requests made to the GraphQL \ac{API}. As before, there is a massive difference in the size of the responses and the requests. The results are shown in Table \ref{table:results:size-comparison-first-path-no-cache-no-reduction-cache-reduction}. Just as before, there is a difference of 11 GraphQL queries that are sent to the GraphQL \ac{API}. However, due to the reduction of queries, the difference in the size of the requests and responses is greater than in Section \ref{subsection:results:comparison-first-second-approach}. All queries of the first approach send 3.93 KB more and return about 2.41 MB more from the GraphQL \ac{API} compared to the third approach. A shared cache and query reduction can save about 22\% response sizes. As before, 37\% fewer records need to be retrieved from the GraphQL \ac{API}.

\ifshowTables
\begin{table}[H]
  \begin{tabular}{|l|l|l|l|l|}
  \hline
  & \textbf{Req. size (B)} & \textbf{Resp. size (B)} & \textbf{Requests} & \textbf{Records}  \\
  \hline
  \textbf{No Reduction, Separate Cache} & 17462 & 10780656 & 47 & 81510 \\
  \hline
  \textbf{Reduction, Shared Cache} & 13533 & 8374763 & 36 & 51319 \\
  \hline
  \hline
  \textbf{Diff (B)} & \textbf{3929} & \textbf{2405893} & \textbf{11} & \textbf{30191} \\
  \hline
  \textbf{Reduction (\%)} & \textbf{23\%} & \textbf{22\%} & \textbf{23\%} & \textbf{37\%} \\
  \hline
  \end{tabular}
  \caption{First Journey: Compare the requests and responses of the first- and third-approach.}\label{table:results:size-comparison-first-path-no-cache-no-reduction-cache-reduction}
\end{table}
\fi

\subsection{Compare the second- and third-approach}\label{subsection:results:comparison-second-third-approach}

Between the second and third approaches, there is almost no difference in request- and response size compared to the comparisons from Sections \ref{subsection:results:comparison-first-second-approach} and \ref{subsection:results:comparison-first-third-approach}, as seen in Table \ref{table:results:size-comparison-first-path-cache-no-reduction-cache-reduction}. Both approaches have the same number of queries sent to the GraphQL \ac{API} since all micro-frontends use the same cache instance. Removing fields from queries does not lead to fewer network requests, because it just removes fields from queries. Network requests would only be omitted if all of the data is already in the cache, but then the query would not be reduced. The difference in request and response size between the two approaches comes solely from query reduction. Using the third approach, the difference in request size is about 1.64 KB (11\%), which is insignificant. The difference between the response sizes (62.45 KB) is almost zero relative to the amount of data returned.

\ifshowTables
\begin{table}[H]
  \begin{tabular}{|l|l|l|l|l|}
  \hline
  & \textbf{Req. size (B)} & \textbf{Resp. size (B)} & \textbf{Requests} & \textbf{Records} \\
  \hline
  \textbf{No Reduction, Shared Cache} & 15176 &  8437211 & 36 & 51319 \\
  \hline
  \textbf{Reduction, Shared Cache} &  13533 &  8374763 & 36 & 51319 \\
  \hline
  \hline
  \textbf{Diff (B)} & \textbf{1643} & \textbf{62448} & \textbf{0} & \textbf{0} \\
  \hline
  \textbf{Reduction (\%)} & \textbf{11\%} & \textbf{1\%} & \textbf{-} & \textbf{-} \\
  \hline
  \end{tabular}
  \caption{First Journey: Compare the requests and responses of the second- and third-approach.}\label{table:results:size-comparison-first-path-cache-no-reduction-cache-reduction}
\end{table}
\fi

\section{Comparing the results of the second user journey}\label{section:results:comparison-second-journey}

This section shows another but more concise comparison of the results between the three approaches explained in \ref{section:results:performance-measurement}. The journey of the client through the application is shown in figure \ref{fig:results:evaluation-second-path}. The client has to perform 17 steps throughout the application, which involves running every available GraphQL query. In contrast to the first journey from \ref{section:results:comparison-first-journey}, the client uses an authenticated user to perform the test. The GraphQL \ac{API} request to retrieve the authenticated user has to be done by every micro-frontend individually, with the default approach with a separate cache.

\ifshowImages
\begin{figure}[H]
\centering
\includegraphics[width=1\linewidth]{images/results/evaluation-second-path.png}
\caption{The second user journey through the application to measure the performance of the micro-frontend architecture.}\label{fig:results:evaluation-second-path}
\end{figure}
\fi

\noindent The next sections compare the three different approaches in terms of request sizes and response sizes, the number of requests, and the total records fetched just like in the previous section \ref{section:results:comparison-first-journey}.

\subsection{Comparing the first- and second-approach}\label{subsection:results:comparison-second-path-first-second-approach}

When comparing the first- with the second approach there is a difference of 25 network requests made to the GraphQL \ac{API} and the size of the requests and responses, as seen in table \ref{table:results:size-comparison-second-path-cache-no-reduction-cache-reduction}. There are no reduced queries for this comparison, the 25 extra requests account for the difference in size. 22\% of the total response size can be saved by using a shared cache layer. Another interesting observation is that the shared cache approach retrieves 30401 fewer records than the naive approach, which is about 37\% of the total records returned.

\ifshowTables
\begin{table}[H]
  \begin{tabular}{|l|l|l|l|l|}
  \hline
  & \textbf{Request Size (B)} & \textbf{Response Size (B)} & \textbf{Requests} & \textbf{Records} \\
  \hline
  \textbf{No Reduction, Separate Cache} & 22955 & 10713304 & 62 & 81325 \\
  \hline
  \textbf{No Reduction, Shared Cache} & 16884 & 8364416 & 37 & 50924 \\
  \hline
  \hline
  \textbf{Diff} & \textbf{6071} & \textbf{2348888} & \textbf{25} & \textbf{30401} \\
  \hline
  \textbf{Reduction (\%)} & \textbf{26\%} & \textbf{22\%} & \textbf{40\%} & \textbf{37\%} \\
  \hline
  \end{tabular}
  \caption{Second Journey: Comparing the requests and responses of the second- and third-approach.}\label{table:results:size-comparison-second-path-cache-no-reduction-cache-reduction}
\end{table}
\fi

\subsection{Comparing the first- and third-approach}\label{subsection:results:comparison-second-path-second-third-approach}

Like the previous comparison, there are again 25 requests less made to the GraphQL \ac{API}. The size of the responses and the requests have about the same difference like before. The results are shown in table \ref{table:results:size-comparison-second-path-no-cache-no-reduction-cache-reduction}. However, due to the reduction in queries, the difference in the size of the queries and responses is a bit greater than in section \ref{table:results:size-comparison-second-path-cache-no-reduction-cache-reduction}. A shared caching layer and query reduction can save about 22\% of response size. As before, 37\% fewer records need to be fetched from the backend.

\ifshowTables
\begin{table}[H]
  \begin{tabular}{|l|l|l|l|l|}
  \hline
  & \textbf{Request Size (B)} & \textbf{Response Size (B)} & \textbf{Requests} & \textbf{Records} \\
  \hline
  \textbf{No Reduction, Separate Cache} & 22955 & 10713304 & 62 & 81325 \\
  \hline
  \textbf{Reduction, Shared Cache} & 14718 & 8361306 & 37 & 50924 \\
  \hline
  \hline
  \textbf{Diff} & \textbf{8237} & \textbf{2351998} & \textbf{25} & \textbf{30401} \\
  \hline
  \textbf{Reduction (\%)} & \textbf{35\%} & \textbf{22\%} & \textbf{40\%} & \textbf{37\%} \\
  \hline
  \end{tabular}
  \caption{Second Journey: Comparing the requests and responses of the first- and third-approach.}\label{table:results:size-comparison-second-path-no-cache-no-reduction-cache-reduction}
\end{table}
\fi

\subsection{Comparing the second- and third-approach}\label{subsection:results:comparison-second-path-first-third-approach}

Between the first- and the second approach, there is almost no difference in terms of request- and response size. The results are displayed table \ref{table:results:size-comparison-first-path-no-cache-no-reduction-cache-reduction}. Both approaches have the same number of queries sent to the GraphQL \ac{API} since the cache is shared by all micro-frontends. The difference in request and response size comes only from using the query reduction mechanism. The difference in request size is 11\%, but they account for just 1.64 KB, which is not significant. The difference between the response sizes (62.45 KB) is almost zero like in the first journey.

\ifshowTables
\begin{table}[H]
\begin{tabular}{|l|l|l|l|l|}
  \hline
  & \textbf{Request Size (B)} & \textbf{Response Size (B)} & \textbf{Requests} & \textbf{Records} \\
  \hline
  \textbf{No Reduction, Shared Cache} & 16884 & 8364416 & 37 & 50924 \\
  \hline
  \textbf{Reduction, Shared Cache} & 14718 & 8361306 & 37 & 50924 \\
  \hline
  \hline
  \textbf{Diff} & \textbf{2166} & \textbf{3110} & \textbf{0} & \textbf{0} \\
  \hline
  \textbf{Reduction (\%)} & \textbf{13\%} & \textbf{0\%} & \textbf{-} & \textbf{-} \\
  \hline
  \end{tabular}
  \caption{Second Journey: Comparing the requests and responses of the first- and second-approach.}\label{table:results:size-comparison-second-path-no-cache-no-reduction-cache-no-reduction}
\end{table}
\fi

\section{Compare original queries and reduced queries}

This section examines the differences in network request size and network response size between the reduced GraphQL queries and the unmodified original queries. The original queries are executed, and the network data is collected. The reduced queries are executed, and the network data is collected. The difference between the original and reduced queries is calculated in percentages and bytes. Table \ref{table:code:comparison-user-reduction} shows the difference in network size for the user-detail query between the original and reduced queries. The query fetches a user by its unique id. The network data was collected by fetching 10 different users from the GraphQL \ac{API}. The requested fields of the user-detail query and which fields are removed fields are explained in more detail in Section \ref{subsection:background:graphql:example-reduction}. The original query contains 16 fields, and 8 fields are removed from the query with the query reduction mechanism. By removing 8 fields from the original GraphQL query, the request size to the GraphQL \ac{API} was reduced by about 30\% or 161 bytes. The request size savings from query reduction is always the same for the requests because the same fields are fetched in every query. When fetching 10 distinct users, a total of 1.61 KB of request size can be saved. The response size was reduced by 42\% on average. About 2.55 KB response size is saved if 10 distinct users are fetched from the GraphQL \ac{API} with the help of query reduction. The response size difference varies slightly from user to user, because the content of the fields is different. The difference from the smallest to the largest user is only about 2\%.

\ifshowTables
\begin{table}[!htbp]
  \begin{tabular}{|l|l|l|l|l|}
  \hline
  \textbf{Query} & \textbf{Req. diff (\%)} & \textbf{Req. size diff (B)} & \textbf{Resp. diff (\%)} & \textbf{Resp. size diff (B)} \\
  \hline
  User A & 30\% & 161 & 42\% & 257 \\
  \hline
  User B & 30\% & 161 & 42\% & 257 \\
  \hline
  User C & 30\% & 161 & 41\% & 257 \\
  \hline
  User D & 30\% & 161 & 42\% & 244 \\
  \hline
  User E & 30\% & 161 & 43\% & 251 \\
  \hline
  User F & 30\% & 161 & 42\% & 271 \\
  \hline
  User G & 30\% & 161 & 42\% & 249 \\
  \hline
  User H & 30\% & 161 & 41\% & 263 \\
  \hline
  User I & 30\% & 161 & 41\% & 248 \\
  \hline
  User J & 30\% & 161 & 41\% & 252 \\
  \hline
  \hline
  \textbf{AVG} & \textbf{30\%} & - & \textbf{42\%} & -  \\
  \hline
  \hline
  \textbf{SUM} & - & \textbf{1610 (1.61 KB)} & - & \textbf{2549 (2.55 KB)} \\
  \hline
  \multicolumn{5}{l}{16 fields requested, 8 fields Removed, 8 remaining sent to the GraphQL \ac{API}}
  \end{tabular}
  \caption{A comparison of the user-detail query in request- and response-sizes.}\label{table:code:comparison-user-reduction}
\end{table}
\fi

% \bigskip

\noindent Table \ref{table:code:comparison-contract-reduction} shows the difference in network size for the contract-detail query between the original and reduced queries. The query fetches the data to display a contract with its unique id. The data was collected by fetching 10 different contracts. The original query contains 17 fields, where 10 fields were removed from the query with the query reduction mechanism. Therefore, only 7 of the 17 fields remain inside the query, which means that 58\% of the fields were removed. The size of the requests to the GraphQL \ac{API} can be reduced by an average of 38\% or 207 bytes. Using the application and querying 10 detail views of a contract 2.07 KB of request size can be saved. The response size from the GraphQL \ac{API} can be reduced by about 58\% or 3.96 KB when fetching 10 different contracts. The response size savings vary slightly from contract to contract because the content of the fields is different. The difference from the smallest to the largest contract is only about 3\%.

\ifshowTables
\begin{table}[!htbp]
  \begin{tabular}{|l|l|l|l|l|}
  \hline
  \textbf{Query} & \textbf{Req. diff (\%)}  & \textbf{Req. size diff (B)} & \textbf{Resp. diff (\%)} & \textbf{Resp. size diff (B)}  \\
  \hline
  Contract A & 38\% & 207 & 58\% & 394 \\
  \hline
  Contract B & 38\% & 207 & 58\% & 378 \\
  \hline
  Contract C & 38\% & 207 & 59\% & 403 \\
  \hline
  Contract D & 38\% & 207 & 60\% & 408 \\
  \hline
  Contract E & 38\% & 207 & 60\% & 409 \\
  \hline
  Contract F & 38\% & 207 & 57\% & 370 \\
  \hline
  Contract G & 38\% & 207 & 58\% & 393 \\
  \hline
  Contract H & 38\% & 207 & 58\% & 407 \\
  \hline
  Contract I  & 38\% & 207 & 59\% & 400 \\
  \hline
  Contract J & 38\% & 207 & 58\% & 389 \\
  \hline
  \hline
  \textbf{AVG} & \textbf{38\%} & - & \textbf{58\%} & - \\
  \hline
  \hline
  \textbf{SUM} & - & \textbf{2070 (2.07 KB)} & - & \textbf{3951 (3.95 KB)} \\
  \hline
  \multicolumn{5}{l}{17 fields requested, 10 fields removed, 7 remaining sent to the GraphQL \ac{API}}
  \end{tabular}
  \caption{A comparison of the contract-detail query in request- and response-sizes.}\label{table:code:comparison-contract-reduction}
\end{table}
\fi
